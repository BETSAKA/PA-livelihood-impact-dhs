% Options for packages loaded elsewhere
\PassOptionsToPackage{unicode}{hyperref}
\PassOptionsToPackage{hyphens}{url}
\PassOptionsToPackage{dvipsnames,svgnames,x11names}{xcolor}
%
\documentclass[
]{article}

\usepackage{amsmath,amssymb}
\usepackage{iftex}
\ifPDFTeX
  \usepackage[T1]{fontenc}
  \usepackage[utf8]{inputenc}
  \usepackage{textcomp} % provide euro and other symbols
\else % if luatex or xetex
  \usepackage{unicode-math}
  \defaultfontfeatures{Scale=MatchLowercase}
  \defaultfontfeatures[\rmfamily]{Ligatures=TeX,Scale=1}
\fi
\usepackage{lmodern}
\ifPDFTeX\else  
    % xetex/luatex font selection
\fi
% Use upquote if available, for straight quotes in verbatim environments
\IfFileExists{upquote.sty}{\usepackage{upquote}}{}
\IfFileExists{microtype.sty}{% use microtype if available
  \usepackage[]{microtype}
  \UseMicrotypeSet[protrusion]{basicmath} % disable protrusion for tt fonts
}{}
\makeatletter
\@ifundefined{KOMAClassName}{% if non-KOMA class
  \IfFileExists{parskip.sty}{%
    \usepackage{parskip}
  }{% else
    \setlength{\parindent}{0pt}
    \setlength{\parskip}{6pt plus 2pt minus 1pt}}
}{% if KOMA class
  \KOMAoptions{parskip=half}}
\makeatother
\usepackage{xcolor}
\setlength{\emergencystretch}{3em} % prevent overfull lines
\setcounter{secnumdepth}{5}
% Make \paragraph and \subparagraph free-standing
\makeatletter
\ifx\paragraph\undefined\else
  \let\oldparagraph\paragraph
  \renewcommand{\paragraph}{
    \@ifstar
      \xxxParagraphStar
      \xxxParagraphNoStar
  }
  \newcommand{\xxxParagraphStar}[1]{\oldparagraph*{#1}\mbox{}}
  \newcommand{\xxxParagraphNoStar}[1]{\oldparagraph{#1}\mbox{}}
\fi
\ifx\subparagraph\undefined\else
  \let\oldsubparagraph\subparagraph
  \renewcommand{\subparagraph}{
    \@ifstar
      \xxxSubParagraphStar
      \xxxSubParagraphNoStar
  }
  \newcommand{\xxxSubParagraphStar}[1]{\oldsubparagraph*{#1}\mbox{}}
  \newcommand{\xxxSubParagraphNoStar}[1]{\oldsubparagraph{#1}\mbox{}}
\fi
\makeatother


\providecommand{\tightlist}{%
  \setlength{\itemsep}{0pt}\setlength{\parskip}{0pt}}\usepackage{longtable,booktabs,array}
\usepackage{calc} % for calculating minipage widths
% Correct order of tables after \paragraph or \subparagraph
\usepackage{etoolbox}
\makeatletter
\patchcmd\longtable{\par}{\if@noskipsec\mbox{}\fi\par}{}{}
\makeatother
% Allow footnotes in longtable head/foot
\IfFileExists{footnotehyper.sty}{\usepackage{footnotehyper}}{\usepackage{footnote}}
\makesavenoteenv{longtable}
\usepackage{graphicx}
\makeatletter
\newsavebox\pandoc@box
\newcommand*\pandocbounded[1]{% scales image to fit in text height/width
  \sbox\pandoc@box{#1}%
  \Gscale@div\@tempa{\textheight}{\dimexpr\ht\pandoc@box+\dp\pandoc@box\relax}%
  \Gscale@div\@tempb{\linewidth}{\wd\pandoc@box}%
  \ifdim\@tempb\p@<\@tempa\p@\let\@tempa\@tempb\fi% select the smaller of both
  \ifdim\@tempa\p@<\p@\scalebox{\@tempa}{\usebox\pandoc@box}%
  \else\usebox{\pandoc@box}%
  \fi%
}
% Set default figure placement to htbp
\def\fps@figure{htbp}
\makeatother
% definitions for citeproc citations
\NewDocumentCommand\citeproctext{}{}
\NewDocumentCommand\citeproc{mm}{%
  \begingroup\def\citeproctext{#2}\cite{#1}\endgroup}
\makeatletter
 % allow citations to break across lines
 \let\@cite@ofmt\@firstofone
 % avoid brackets around text for \cite:
 \def\@biblabel#1{}
 \def\@cite#1#2{{#1\if@tempswa , #2\fi}}
\makeatother
\newlength{\cslhangindent}
\setlength{\cslhangindent}{1.5em}
\newlength{\csllabelwidth}
\setlength{\csllabelwidth}{3em}
\newenvironment{CSLReferences}[2] % #1 hanging-indent, #2 entry-spacing
 {\begin{list}{}{%
  \setlength{\itemindent}{0pt}
  \setlength{\leftmargin}{0pt}
  \setlength{\parsep}{0pt}
  % turn on hanging indent if param 1 is 1
  \ifodd #1
   \setlength{\leftmargin}{\cslhangindent}
   \setlength{\itemindent}{-1\cslhangindent}
  \fi
  % set entry spacing
  \setlength{\itemsep}{#2\baselineskip}}}
 {\end{list}}
\usepackage{calc}
\newcommand{\CSLBlock}[1]{\hfill\break\parbox[t]{\linewidth}{\strut\ignorespaces#1\strut}}
\newcommand{\CSLLeftMargin}[1]{\parbox[t]{\csllabelwidth}{\strut#1\strut}}
\newcommand{\CSLRightInline}[1]{\parbox[t]{\linewidth - \csllabelwidth}{\strut#1\strut}}
\newcommand{\CSLIndent}[1]{\hspace{\cslhangindent}#1}

\usepackage{times} % Police Times New Roman
\usepackage{setspace} % Pour interligne
\onehalfspacing % Interligne de 1.5
\usepackage[colorlinks=true,linkcolor=blue,urlcolor=blue]{hyperref} % Activation des liens cliquable

%Packages pour les figures
\usepackage{graphicx} %Inclusion d'images
\usepackage{float} %Positionnement des figures
\usepackage{caption} %Personnalisation des légendes
\usepackage{subcaption} %Gestion des sous-figures

\renewcommand{\maketitle}{
  \begin{center}
    \textbf{Journal of Development Economics} \\
    \textbf{Registered Report Stage 1 : Proposal} \\[1em]
    {\Huge \textbf{Conservation and development: Socioeconomic Impact evaluation of Terrestrial Protected Areas in Madagascar based on large national surveys}} \\[1em]
    \Large Iriana Razafimahenina\footnotemark[1]\footnotemark[2]\footnotemark[3]\footnotemark[4],
    Florent Bédécarrats\footnotemark[5]\footnotemark[6],
    Ingrid Dallmann\footnotemark[7],
    Holimalala Randriamanampisoa\footnotemark[1]\footnotemark[3]\footnotemark[6] \\
  \end{center}
  \footnotetext[1]{\href{https://www.univ-antananarivo.mg}{University of Antananarivo}, Madagascar}
  \footnotetext[2]{\href{https://www.universite-paris-saclay.fr}{University of Paris Saclay}, France}
  \footnotetext[3]{Development Centre for Economic Studies and Research (CERED), Madagascar}
  \footnotetext[4]{\href{https://www.ird.fr}{French National Research Institute for Sustainable Development (IRD)}, Madagascar}
  \footnotetext[5]{\href{https://www.uvsq.fr}{University of Saint Quentin en Yvelines}, France}
  \footnotetext[6]{\href{https://www.umi-source.uvsq.fr/}{UMI - Sustainability and Resilience (SOURCE)}, IRD, France}
  \footnotetext[7]{\href{https://www.afd.fr}{Agence Française de Développement (AFD)}, France}
}
\usepackage{sectsty}
\usepackage{titlesec}
\newcommand{\mycenteredsection}[1]{\section*{\centering #1}}
\usepackage{geometry}
\geometry{margin=1.2in}
\usepackage{ragged2e}
\justifying
\interfootnotelinepenalty=10000
\makeatletter
\@ifpackageloaded{caption}{}{\usepackage{caption}}
\AtBeginDocument{%
\ifdefined\contentsname
  \renewcommand*\contentsname{Table of contents}
\else
  \newcommand\contentsname{Table of contents}
\fi
\ifdefined\listfigurename
  \renewcommand*\listfigurename{List of Figures}
\else
  \newcommand\listfigurename{List of Figures}
\fi
\ifdefined\listtablename
  \renewcommand*\listtablename{List of Tables}
\else
  \newcommand\listtablename{List of Tables}
\fi
\ifdefined\figurename
  \renewcommand*\figurename{Figure}
\else
  \newcommand\figurename{Figure}
\fi
\ifdefined\tablename
  \renewcommand*\tablename{Table}
\else
  \newcommand\tablename{Table}
\fi
}
\@ifpackageloaded{float}{}{\usepackage{float}}
\floatstyle{ruled}
\@ifundefined{c@chapter}{\newfloat{codelisting}{h}{lop}}{\newfloat{codelisting}{h}{lop}[chapter]}
\floatname{codelisting}{Listing}
\newcommand*\listoflistings{\listof{codelisting}{List of Listings}}
\makeatother
\makeatletter
\makeatother
\makeatletter
\@ifpackageloaded{caption}{}{\usepackage{caption}}
\@ifpackageloaded{subcaption}{}{\usepackage{subcaption}}
\makeatother

\usepackage{bookmark}

\IfFileExists{xurl.sty}{\usepackage{xurl}}{} % add URL line breaks if available
\urlstyle{same} % disable monospaced font for URLs
\hypersetup{
  pdftitle={Conservation and development: Socioeconomic Impact evaluation of Terrestrial Protected Areas in Madagascar based on large national surveys},
  pdfauthor={Iriana Razafimahenina; Florent Bédécarrats; Ingrid Dallmann; Holimalala Randriamanampisoa},
  colorlinks=true,
  linkcolor={blue},
  filecolor={Maroon},
  citecolor={Blue},
  urlcolor={Blue},
  pdfcreator={LaTeX via pandoc}}


\title{Conservation and development: Socioeconomic Impact evaluation of
Terrestrial Protected Areas in Madagascar based on large national
surveys}
\author{Iriana
Razafimahenina\footnotemark[1]\footnotemark[2]\footnotemark[3]\footnotemark[4] \and Florent
Bédécarrats\footnotemark[5]\footnotemark[6] \and Ingrid
Dallmann\footnotemark[7] \and Holimalala
Randriamanampisoa\footnotemark[1]\footnotemark[3]\footnotemark[6]}
\date{}

\begin{document}
\maketitle


\mycenteredsection{Date of latest draft: 04/12/2024}

\mycenteredsection{Abstract}

Protected Areas are the most widely used tool for biodiversity
conservation. However, their implementation raises concerns about the
well-being of local populations, especially when they are very poor and
dependent on natural resources, as is the case in Madagascar. This
pre-analysis plan outlines the data, methods, and empirical strategies
used to evaluate the impact of protected areas on local household
well-being and the inequalities among them. Our study focuses on
terrestrial protected areas and relies on Demographic Health Surveys
spanning a 13-years period (2008-2021). We will also use data from the
previous 11 years (1997-2008) to assess whether parallel trends prior to
the study period confirm the validity of the comparisons. The data will
be analyzed using spatio-temporal models, matching, and
difference-in-differences methods.

\textbf{Keywords} : Biodiversity Conservation, Well-Being, Demographic
and Health Surveys, Spatio-Temporal Models, Geospatial impact
evaluation, Madagascar

JEL codes : Q57, I31, C31, Q56, O55

\textbf{Study pre-registration}: On open Science Framework (OSF) with
the title `Conservation and development: socioeconomic impact evaluation
of terrestrial Protected Areas in Madagascar based on large national
surveys' \href{https://osf.io/bgu5n/}{https://osf.io/bgu5n/}

\newpage

\textbf{Proposed timeline}

\begin{longtable}[]{@{}
  >{\raggedright\arraybackslash}p{(\linewidth - 2\tabcolsep) * \real{0.6000}}
  >{\raggedright\arraybackslash}p{(\linewidth - 2\tabcolsep) * \real{0.4000}}@{}}
\toprule\noalign{}
\begin{minipage}[b]{\linewidth}\raggedright
Phases
\end{minipage} & \begin{minipage}[b]{\linewidth}\raggedright
Dates
\end{minipage} \\
\midrule\noalign{}
\endhead
\bottomrule\noalign{}
\endlastfoot
Literature Review, Conceptualization, and writing of the Registered
report & May 2024 - January 2025 \\
Retrieve data from selected sources & February 2025 \\
Data cleaning and analysis & February - March 2025 \\
Writing the scientific article & March 2025 - April 2025 \\
Submission to the journal & April 2025 \\
\end{longtable}

\newpage

\section{Introduction}\label{introduction}

The reconciliation between conservation and development has been a
long-discussed issue within the scientific community (Adams et al.,
2004) , but its importance has grown considerably over the past decade
with the rapid expansion of protected areas (PAs). This issue is
particularly relevant for all 195 COP15 signatory states, which have
committed to increasing protected areas coverage to 30\% of terrestrial
land by 2030.

In theory, protected areas can have significant impacts on local
livelihoods, both positive and negative. They are recognized as an
essential tool for biodiversity conservation (Maxwell et al., 2020), but
their creation can deprive nearby communities of access to natural
resources (gathering, hunting, fishing, and medicinal plants), reduce
the amount of land available and restrict economic activities
(agriculture, livestock, construction) (Kandel et al., 2022).
Conversely, they can be accompanied by compensation measures (local
development projects, cash transfers), generate economic benefits (jobs
in protected areas, tourism), and enhance ecosystem services (increased
water resources, erosion control, fire prevention) (Kandel et al.,
2022).

Despite these ambivalent potential effects, empirical studies that
rigorously assess the impact of protected areas on people's livelihoods
are still rare. Of the 1,043 studies reviewed by McKinnon et al. (2016)
, only 19 used quantitative methods to evaluate impacts on material
living conditions or economic well-being. This meta-analysis shows that
the results of studies vary widely depending on the methods used, the
context studied, and the location. Kandel et al. (2022) have updated and
extended this analysis by focusing on a corpus of 30 quantitative
evaluations that specifically address the impact of protected areas on
household income. They show that protected areas can have a positive
impact on local economies, but that this effect is generally modest and
depends on the local context. This variability in impacts highlights the
importance of conducting context-specific studies using robust
quantitative methods.

Madagascar stands out as a particularly relevant case study for
analyzing the relationship between conservation and socioeconomic
conditions. The country is the poorest in terms of the first target of
the Sustainable Development Goals (SDG 1-1), with the highest proportion
of the population living below the international poverty line in the
world (Conceição (2024), p.~298-99). In 2008, terrestrial protected
areas covered 3.6\% of Madagascar and 9\% of the population lived within
10 km of a protected area. Today, they cover 10.8\% and 28\% of the
population live within 10 km of protected
areas\textsuperscript{\hyperref[_ftn1]{{[}1{]}}}. Madagascar is also
characterized by a low state capacity (Hanson \& Sigman, 2021), which
makes it difficult to implement conservation and sustainable development
policies and the social measures that should accompany them. These
factors, combined with the high dependence of the rural population on
natural resources, mean that the impacts of protected areas are
potentially different from those observed in less precarious contexts.

However, empirical studies at the national scale are almost non-existent
for Madagascar. None of the quantitative impact evaluations identified
by McKinnon et al. (2016) covered the country. One of the references
consolidated by Kandel et al. (2022) is a multi-country study that
includes Madagascar, but it is based on an estimate of an aggregate
impact at the commune level and covers only one date. It uses the 1993
census data to match the country's municipalities (Mammides, 2020),
without a before-and-after comparison, and in a context where less than
3~\% of the territory was covered by protected areas, most of which had
been created several decades earlier.

Our contribution to the literature is twofold, both empirical and
methodological. Empirically, this study provides an unprecedented
national analysis, covering 71 protected areas established between 2008
and 2021, to evaluate the socioeconomic impacts of conservation in
contexts of extreme poverty and weak governance. Methodologically, it
articulates the state of the art in econometrics, incorporating recent
developments to adapt these methods to the study of protected areas. The
procedure we propose here could be replicated in other countries,
starting with the 39 countries that have at least three geolocated DHS
surveys. This approach paves the way for a more systematic evaluation of
the impact of protected areas, taking into account the specific context
of each country.

\hyperref[_ftnref1]{{[}1{]}}~~~~ Calculations by the authors based on
the location of the DHS survey clusters. The detailed calculation is
provided as supplementary material to the study.

\section{Research Design}\label{research-design}

\subsection{Hypothesis}\label{hypothesis}

Our first hypothesis concerns the overall impact of protected areas in
the Malagasy context. In their meta-analysis of 30 studies, Kandel et
al. (2022) report a slightly positive average impact, but highlight a
large heterogeneity of results across context. Several parameters are
likely to influence impact, as represented graphically in
\autoref{fig:logic_diagram} in the form of directed acyclic graph
(Hünermund \& Bareinboim, 2023; Imbens, 2024)

\begin{figure}[H]
    \centering %Centre l'image horizontalement sur la page
    \includegraphics[width=0.8\linewidth]{C:/Users/irian/Documents/Statistiques/PA-livelihood-impact-dhs2/Pre-analysis-plan/images/figure 1_WU_21.png} %contrôle de la taille
    \caption{Logic diagram of the theory of change tested in the study }
    \caption*{\textit{Source:Authors}} %légende
    \label{fig:logic_diagram} %référencement dans le texte
\end{figure}

The factors likely to lead to a decline in well-being seem particularly
significant in the Malagasy context, where the population is
predominantly rural and living in extreme poverty (the last assessment
was in 2012, with 80.7\% of the population below the \$2.15 a day
threshold at 2017 PPP). Six studies conducted in Madagascar between 1995
and 2006 estimated the opportunity cost of losing access to protected
areas (slash-and-burn agriculture, hunting, gathering, timber, etc.) at
between USD 39 and 177 per household per year (Neudert et al., 2017).
Golden et al. (2014) estimated that income from hunting accounted for
57~\% of household cash income in areas adjacent to the Makira and
Masoala protected areas. Another survey of people living near Makira
estimated the value of pharmaceutical use at USD 30-44 per year per
household, based on the subsidized price of equivalent treatments in the
malagasy market (Golden et al., 2012).

Several factors that could help improve livelihoods through conservation
appear to be fragile in Madagascar, starting with tourism. Naidoo et al.
(2019) aggregate data from DHS surveys conducted between 2001 and 2011
in 34 developing countries. Their study is based on matching households
near and far from protected areas, but with no pre-post conservation
comparison. They highlight positive impacts, but only for a subset of
protected areas `with documented
tourism'\textsuperscript{\hyperref[sdfootnote1sym]{1}}. According to
their study, households living near the protected areas `with
tourism'~are 17\% wealthier and 16\% less likely to be poor than similar
households living far from these areas.

\hyperref[sdfootnote1anc]{1}The source used to consider that va PA has
`documented tourism' is not reported in Naidoo et al 2019.

\phantomsection\label{refs}
\begin{CSLReferences}{1}{0}
\bibitem[\citeproctext]{ref-adams_biodiversity_2004}
Adams, William. M., Aveling, R., Brockington, D., Dickson, B., Elliott,
J., Hutton, J., Roe, D., Vira, B., \& Wolmer, W. (2004). Biodiversity
{Conservation} and the {Eradication} of {Poverty}. \emph{Science},
\emph{306}(5699), 1146--1149.
\url{https://doi.org/10.1126/science.1097920}

\bibitem[\citeproctext]{ref-conceicao_breaking_2024}
Conceição, P. (Ed.). (2024). \emph{Breaking the gridlock: {Reimagining}
cooperation in a {Polarized} world}. UNDP.

\bibitem[\citeproctext]{ref-golden_economic_2014}
Golden, C. D., Bonds, M. H., Brashares, J. S., Rodolph Rasolofoniaina,
B. J., \& Kremen, C. (2014). Economic valuation of subsistence harvest
of wildlife in {Madagascar}. \emph{Conservation Biology}, \emph{28}(1),
234--243. https://doi.org/\url{https://doi.org/10.1111/cobi.12174}

\bibitem[\citeproctext]{ref-golden_rainforest_2012}
Golden, C. D., Rasolofoniaina, B. R., Anjaranirina, E. G., Nicolas, L.,
Ravaoliny, L., \& Kremen, C. (2012). \emph{Rainforest pharmacopeia in
{Madagascar} provides high value for current local and prospective
global uses}.

\bibitem[\citeproctext]{ref-hanson_leviathans_2021}
Hanson, J. K., \& Sigman, R. (2021). Leviathan's {Latent} {Dimensions}:
{Measuring} {State} {Capacity} for {Comparative} {Political} {Research}.
\emph{The Journal of Politics}, \emph{83}(4), 1495--1510.
\url{https://doi.org/10.1086/715066}

\bibitem[\citeproctext]{ref-hunermund_causal_2023}
Hünermund, P., \& Bareinboim, E. (2023). Causal inference and data
fusion in econometrics. \emph{The Econometrics Journal}, utad008.

\bibitem[\citeproctext]{ref-imbens_causal_2024}
Imbens, G. W. (2024). Causal {Inference} in the {Social} {Sciences}.
\emph{Annual Review of Statistics and Its Application}, \emph{11}(Volume
11, 2024), 123--152.
\url{https://doi.org/10.1146/annurev-statistics-033121-114601}

\bibitem[\citeproctext]{ref-kandel_protected_2022}
Kandel, P., Pandit, R., White, B., \& Polyakov, M. (2022). Do protected
areas increase household income? {Evidence} from a {Meta}-{Analysis}.
\emph{World Development}, \emph{159}, 106024.
\url{https://doi.org/10.1016/j.worlddev.2022.106024}

\bibitem[\citeproctext]{ref-mammides_evidence_2020}
Mammides, C. (2020). Evidence from eleven countries in four continents
suggests that protected areas are not associated with higher poverty
rates. \emph{Biological Conservation}, \emph{241}, 108353.
\url{https://www.sciencedirect.com/science/article/pii/S0006320719312777?casa_token=1saHx-9SppkAAAAA:sw9KzbZvzqu2WLub5u-K06mA2kgTygSvTi5AEsjBz0rUm8h3h9SKsdId52pG5VEr4SobaFTfguA}

\bibitem[\citeproctext]{ref-maxwell_area-based_2020}
Maxwell, S. L., Cazalis, V., Dudley, N., Hoffmann, M., Rodrigues, A. S.,
Stolton, S., Visconti, P., Woodley, S., Kingston, N., \& Lewis, E.
(2020). Area-based conservation in the twenty-first century.
\emph{Nature}, \emph{586}(7828), 217--227.

\bibitem[\citeproctext]{ref-mckinnon_what_2016}
McKinnon, M. C., Cheng, S. H., Dupre, S., Edmond, J., Garside, R., Glew,
L., Holland, M. B., Levine, E., Masuda, Y. J., Miller, D. C., Oliveira,
I., Revenaz, J., Roe, D., Shamer, S., Wilkie, D., Wongbusarakum, S., \&
Woodhouse, E. (2016). What are the effects of nature conservation on
human well-being? {A} systematic map of empirical evidence from
developing countries. \emph{Environ Evid}, \emph{5}(1), 8.
\url{https://doi.org/10.1186/s13750-016-0058-7}

\bibitem[\citeproctext]{ref-naidoo_evaluating_2019}
Naidoo, R., Gerkey, D., Hole, D., Pfaff, A., Ellis, A. M., Golden, C.
D., Herrera, D., Johnson, K., Mulligan, M., Ricketts, T. H., \& Fisher,
B. (2019). Evaluating the impacts of protected areas on human well-being
across the developing world. \emph{Science Advances}, \emph{5}(4),
eaav3006. \url{https://doi.org/10.1126/sciadv.aav3006}

\bibitem[\citeproctext]{ref-neudert_global_2017}
Neudert, R., Ganzhorn, J. U., \& Waetzold, F. (2017). Global benefits
and local costs--{The} dilemma of tropical forest conservation: {A}
review of the situation in {Madagascar}. \emph{Environmental
Conservation}, \emph{44}(1), 82--96.

\end{CSLReferences}




\end{document}
