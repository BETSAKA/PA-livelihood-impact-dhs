% Options for packages loaded elsewhere
\PassOptionsToPackage{unicode}{hyperref}
\PassOptionsToPackage{hyphens}{url}
\PassOptionsToPackage{dvipsnames,svgnames,x11names}{xcolor}
%
\documentclass[
]{article}

\usepackage{amsmath,amssymb}
\usepackage{iftex}
\ifPDFTeX
  \usepackage[T1]{fontenc}
  \usepackage[utf8]{inputenc}
  \usepackage{textcomp} % provide euro and other symbols
\else % if luatex or xetex
  \usepackage{unicode-math}
  \defaultfontfeatures{Scale=MatchLowercase}
  \defaultfontfeatures[\rmfamily]{Ligatures=TeX,Scale=1}
\fi
\usepackage{lmodern}
\ifPDFTeX\else  
    % xetex/luatex font selection
\fi
% Use upquote if available, for straight quotes in verbatim environments
\IfFileExists{upquote.sty}{\usepackage{upquote}}{}
\IfFileExists{microtype.sty}{% use microtype if available
  \usepackage[]{microtype}
  \UseMicrotypeSet[protrusion]{basicmath} % disable protrusion for tt fonts
}{}
\makeatletter
\@ifundefined{KOMAClassName}{% if non-KOMA class
  \IfFileExists{parskip.sty}{%
    \usepackage{parskip}
  }{% else
    \setlength{\parindent}{0pt}
    \setlength{\parskip}{6pt plus 2pt minus 1pt}}
}{% if KOMA class
  \KOMAoptions{parskip=half}}
\makeatother
\usepackage{xcolor}
\ifLuaTeX
  \usepackage{luacolor}
  \usepackage[soul]{lua-ul}
\else
  \usepackage{soul}
  
\fi
\setlength{\emergencystretch}{3em} % prevent overfull lines
\setcounter{secnumdepth}{5}
% Make \paragraph and \subparagraph free-standing
\makeatletter
\ifx\paragraph\undefined\else
  \let\oldparagraph\paragraph
  \renewcommand{\paragraph}{
    \@ifstar
      \xxxParagraphStar
      \xxxParagraphNoStar
  }
  \newcommand{\xxxParagraphStar}[1]{\oldparagraph*{#1}\mbox{}}
  \newcommand{\xxxParagraphNoStar}[1]{\oldparagraph{#1}\mbox{}}
\fi
\ifx\subparagraph\undefined\else
  \let\oldsubparagraph\subparagraph
  \renewcommand{\subparagraph}{
    \@ifstar
      \xxxSubParagraphStar
      \xxxSubParagraphNoStar
  }
  \newcommand{\xxxSubParagraphStar}[1]{\oldsubparagraph*{#1}\mbox{}}
  \newcommand{\xxxSubParagraphNoStar}[1]{\oldsubparagraph{#1}\mbox{}}
\fi
\makeatother


\providecommand{\tightlist}{%
  \setlength{\itemsep}{0pt}\setlength{\parskip}{0pt}}\usepackage{longtable,booktabs,array}
\usepackage{calc} % for calculating minipage widths
% Correct order of tables after \paragraph or \subparagraph
\usepackage{etoolbox}
\makeatletter
\patchcmd\longtable{\par}{\if@noskipsec\mbox{}\fi\par}{}{}
\makeatother
% Allow footnotes in longtable head/foot
\IfFileExists{footnotehyper.sty}{\usepackage{footnotehyper}}{\usepackage{footnote}}
\makesavenoteenv{longtable}
\usepackage{graphicx}
\makeatletter
\newsavebox\pandoc@box
\newcommand*\pandocbounded[1]{% scales image to fit in text height/width
  \sbox\pandoc@box{#1}%
  \Gscale@div\@tempa{\textheight}{\dimexpr\ht\pandoc@box+\dp\pandoc@box\relax}%
  \Gscale@div\@tempb{\linewidth}{\wd\pandoc@box}%
  \ifdim\@tempb\p@<\@tempa\p@\let\@tempa\@tempb\fi% select the smaller of both
  \ifdim\@tempa\p@<\p@\scalebox{\@tempa}{\usebox\pandoc@box}%
  \else\usebox{\pandoc@box}%
  \fi%
}
% Set default figure placement to htbp
\def\fps@figure{htbp}
\makeatother
% definitions for citeproc citations
\NewDocumentCommand\citeproctext{}{}
\NewDocumentCommand\citeproc{mm}{%
  \begingroup\def\citeproctext{#2}\cite{#1}\endgroup}
\makeatletter
 % allow citations to break across lines
 \let\@cite@ofmt\@firstofone
 % avoid brackets around text for \cite:
 \def\@biblabel#1{}
 \def\@cite#1#2{{#1\if@tempswa , #2\fi}}
\makeatother
\newlength{\cslhangindent}
\setlength{\cslhangindent}{1.5em}
\newlength{\csllabelwidth}
\setlength{\csllabelwidth}{3em}
\newenvironment{CSLReferences}[2] % #1 hanging-indent, #2 entry-spacing
 {\begin{list}{}{%
  \setlength{\itemindent}{0pt}
  \setlength{\leftmargin}{0pt}
  \setlength{\parsep}{0pt}
  % turn on hanging indent if param 1 is 1
  \ifodd #1
   \setlength{\leftmargin}{\cslhangindent}
   \setlength{\itemindent}{-1\cslhangindent}
  \fi
  % set entry spacing
  \setlength{\itemsep}{#2\baselineskip}}}
 {\end{list}}
\usepackage{calc}
\newcommand{\CSLBlock}[1]{\hfill\break\parbox[t]{\linewidth}{\strut\ignorespaces#1\strut}}
\newcommand{\CSLLeftMargin}[1]{\parbox[t]{\csllabelwidth}{\strut#1\strut}}
\newcommand{\CSLRightInline}[1]{\parbox[t]{\linewidth - \csllabelwidth}{\strut#1\strut}}
\newcommand{\CSLIndent}[1]{\hspace{\cslhangindent}#1}

% Éviter les conflits de définition de commandes
 \let\arrowvert\undefined
 \let\Arrowvert\undefined

 % Encodage et police
 \usepackage[T1]{fontenc}
 \usepackage{newtxtext,newtxmath} % Police Times New Roman
 \usepackage{setspace} % Pour interligne
 \onehalfspacing % Interligne de 1.5
 \usepackage[colorlinks=true,linkcolor=blue, citecolor=blue,urlcolor=blue]{hyperref} % Activation des liens cliquable

 % Packages pour les figures
 \usepackage{graphicx} %Inclusion d'images
 \usepackage{float} %Positionnement des figures
 \usepackage{caption} %Personnalisation des légendes
 \usepackage{subcaption} %Gestion des sous-figures

 % Affichage en mode paysage d'une page de la section Appendix
 \usepackage{pdflscape}  % Pour afficher en paysage
 \usepackage{longtable}  % Permet les tableaux longs et multi-pages
 \usepackage{booktabs}   % Pour des tableaux propres
 \usepackage{array}      % Amélioration des tableaux
 \usepackage{lscape}     % Mode paysage
 \usepackage{tabularx}   % Colonnes auto-ajustées

 % Insertion des tableaux
 \usepackage{multirow} % Ajout du package pour multirow
 \usepackage{booktabs} % Amélioration du rendu des tableaux
 \usepackage{siunitx}  % Pour formater les nombres (espaces entre milliers)
 \usepackage{lmodern}  % Pour des polices plus nettes


 % Personnalisation des notes de bas de page
 \usepackage{footmisc} %Options avancées pour les notes de bas de page
 \renewcommand{\footnoterule}{\noindent\rule{0.3\linewidth}{0.4pt}\vspace{0.2cm}} %Ligne de séparation plus courte
 \renewcommand{\thefootnote}{\textsuperscript{\arabic{footnote}}} % Exposant pour les numéros de notes
 \interfootnotelinepenalty=1000 %Empêcher le découpage des notes sur plusieurs pages


 \renewcommand{\maketitle}{
   \begin{center}
     \textbf{Journal of Development Economics} \\
     \textbf{Registered Report Stage 1 : Proposal} \\[1em]
     {\Huge \textbf{Conservation and development: Socioeconomic Impact evaluation of Terrestrial Protected Areas in Madagascar based on large national surveys}} \\[1em]
     \Large Iriana Razafimahenina\footnotemark[1]\footnotemark[2]\footnotemark[3]\footnotemark[4]\footnotemark[5]\footnotemark[6],
     Florent Bédécarrats\footnotemark[5]\footnotemark[6],
     Ingrid Dallmann\footnotemark[7],
     Holimalala Randriamanampisoa\footnotemark[1]\footnotemark[3]\footnotemark[6] \\
   \end{center}
   \footnotetext[1]{\href{https://www.univ-antananarivo.mg}{University of Antananarivo}, Madagascar}
   \footnotetext[2]{\href{https://www.universite-paris-saclay.fr}{University of Paris Saclay}, France}
   \footnotetext[3]{Development Centre for Economic Studies and Research (CERED), Madagascar}
   \footnotetext[4]{\href{https://www.ird.fr}{French National Research Institute for Sustainable Development (IRD)}, Madagascar}
   \footnotetext[5]{\href{https://www.uvsq.fr}{University of Saint Quentin en Yvelines}, France}
   \footnotetext[6]{\href{https://www.umi-source.uvsq.fr/}{UMI - Sustainability and Resilience (SOURCE)}, IRD, France}
   \footnotetext[7]{\href{https://www.afd.fr}{Agence Française de Développement (AFD)}, France}
 }
 \usepackage{sectsty}
 \usepackage{titlesec}
 \newcommand{\mycenteredsection}[1]{\section*{\centering #1}}
 \usepackage{geometry}
 \geometry{margin=1.2in}
 \usepackage{ragged2e}
 \justifying
 \interfootnotelinepenalty=10000
\makeatletter
\@ifpackageloaded{caption}{}{\usepackage{caption}}
\AtBeginDocument{%
\ifdefined\contentsname
  \renewcommand*\contentsname{Table of contents}
\else
  \newcommand\contentsname{Table of contents}
\fi
\ifdefined\listfigurename
  \renewcommand*\listfigurename{List of Figures}
\else
  \newcommand\listfigurename{List of Figures}
\fi
\ifdefined\listtablename
  \renewcommand*\listtablename{List of Tables}
\else
  \newcommand\listtablename{List of Tables}
\fi
\ifdefined\figurename
  \renewcommand*\figurename{Figure}
\else
  \newcommand\figurename{Figure}
\fi
\ifdefined\tablename
  \renewcommand*\tablename{Table}
\else
  \newcommand\tablename{Table}
\fi
}
\@ifpackageloaded{float}{}{\usepackage{float}}
\floatstyle{ruled}
\@ifundefined{c@chapter}{\newfloat{codelisting}{h}{lop}}{\newfloat{codelisting}{h}{lop}[chapter]}
\floatname{codelisting}{Listing}
\newcommand*\listoflistings{\listof{codelisting}{List of Listings}}
\makeatother
\makeatletter
\makeatother
\makeatletter
\@ifpackageloaded{caption}{}{\usepackage{caption}}
\@ifpackageloaded{subcaption}{}{\usepackage{subcaption}}
\makeatother

\usepackage{bookmark}

\IfFileExists{xurl.sty}{\usepackage{xurl}}{} % add URL line breaks if available
\urlstyle{same} % disable monospaced font for URLs
\hypersetup{
  pdftitle={Conservation and development: Socioeconomic Impact evaluation of Terrestrial Protected Areas in Madagascar based on large national surveys},
  pdfauthor={Iriana Razafimahenina; Florent Bédécarrats; Ingrid Dallmann; Holimalala Randriamanampisoa},
  colorlinks=true,
  linkcolor={blue},
  filecolor={Maroon},
  citecolor={Blue},
  urlcolor={Blue},
  pdfcreator={LaTeX via pandoc}}


\title{Conservation and development: Socioeconomic Impact evaluation of
Terrestrial Protected Areas in Madagascar based on large national
surveys}
\author{Iriana
Razafimahenina\footnotemark[1]\footnotemark[2]\footnotemark[3]\footnotemark[4]\footnotemark[5]\footnotemark[6] \and Florent
Bédécarrats\footnotemark[5]\footnotemark[6] \and Ingrid
Dallmann\footnotemark[7] \and Holimalala
Randriamanampisoa\footnotemark[1]\footnotemark[3]\footnotemark[6]}
\date{}

\begin{document}
\maketitle


\mycenteredsection{Date of latest draft: 04/12/2024}

\mycenteredsection{Abstract}

Protected Areas are the most widely used tool for biodiversity
conservation. However, their implementation raises concerns about the
well-being of local populations, especially when they are very poor and
dependent on natural resources, as is the case in Madagascar. This
pre-analysis plan outlines the data, methods, and empirical strategies
used to evaluate the impact of protected areas on local household
well-being and the inequalities among them. Our study focuses on
terrestrial protected areas and relies on Demographic Health Surveys
spanning a 13-years period (2008-2021). We will also use data from the
previous 11 years (1997-2008) to assess whether parallel trends prior to
the study period confirm the validity of the comparisons. The data will
be analyzed using spatio-temporal models, matching, and
difference-in-differences methods.

\textbf{Keywords} : Biodiversity Conservation, Well-Being, Demographic
and Health Surveys, Spatio-Temporal Models, Geospatial impact
evaluation, Madagascar

JEL codes : Q57, I31, C31, Q56, O55

\textbf{Study pre-registration}: On open Science Framework (OSF) with
the title `Conservation and development: socioeconomic impact evaluation
of terrestrial Protected Areas in Madagascar based on large national
surveys' \href{https://osf.io/bgu5n/}{https://osf.io/bgu5n/}

\newpage

\textbf{Proposed timeline}

\begin{longtable}[]{@{}
  >{\raggedright\arraybackslash}p{(\linewidth - 2\tabcolsep) * \real{0.5000}}
  >{\raggedright\arraybackslash}p{(\linewidth - 2\tabcolsep) * \real{0.5000}}@{}}
\toprule\noalign{}
\begin{minipage}[b]{\linewidth}\raggedright
Phases
\end{minipage} & \begin{minipage}[b]{\linewidth}\raggedright
Dates
\end{minipage} \\
\midrule\noalign{}
\endhead
\bottomrule\noalign{}
\endlastfoot
Literature Review, Conceptualization, and writing of the Registered
report & May 2024 - January 2025 \\
Retrieve data from selected sources & February 2025 \\
Data cleaning and analysis & February - March 2025 \\
Writing the scientific article & March 2025 - April 2025 \\
Submission to the journal & April 2025 \\
\end{longtable}

\newpage

\section{\texorpdfstring{\textbf{Introduction}}{Introduction}}\label{introduction}

The reconciliation between conservation and development has been a
long-discussed issue within the scientific community The reconciliation
between conservation and development has been a long-discussed issue
within the scientific community (Adams et al., 2004), but its importance
has grown considerably over the past decade with the rapid expansion of
protected areas (PAs). This issue is particularly relevant for all 195
COP15 signatory states, which have committed to increasing protected
areas coverage to 30\% of terrestrial land by 2030.

In theory, protected areas can have significant impacts on local
livelihoods, both positive and negative. They are recognized as an
essential tool for biodiversity conservation (Maxwell et al., 2020), but
their creation can deprive nearby communities of access to natural
resources (gathering, hunting, fishing, and medicinal plants), reduce
the amount of land available and restrict economic activities
(agriculture, livestock, construction) (Kandel et al., 2022).
Conversely, they can be accompanied by compensation measures (local
development projects, cash transfers), generate economic benefits (jobs
in protected areas, tourism), and enhance ecosystem services (increased
water resources, erosion control, fire prevention) (Kandel et al.,
2022).

Despite these ambivalent potential effects, empirical studies that
rigorously assess the impact of protected areas on people's livelihoods
are still rare. Of the 1,043 studies reviewed by McKinnon et al. (2016)
, only 19 used quantitative methods to evaluate impacts on material
living conditions or economic well-being. This meta-analysis shows that
the results of studies vary widely depending on the methods used, the
context studied, and the location. Kandel et al. (2022) have updated and
extended this analysis by focusing on a corpus of 30 quantitative
evaluations that specifically address the impact of protected areas on
household income. They show that protected areas can have a positive
impact on local economies, but that this effect is generally modest and
depends on the local context. This variability in impacts highlights the
importance of conducting context-specific studies using robust
quantitative methods.

Madagascar stands out as a particularly relevant case study for
analyzing the relationship between conservation and socioeconomic
conditions. The country is the poorest in terms of the first target of
the Sustainable Development Goals (SDG 1-1), with the highest proportion
of the population living below the international poverty line in the
world (Conceição (2024), p.~298-99). In 2008, terrestrial protected
areas covered 3.6\% of Madagascar and 9\% of the population lived within
10 km of a protected area. Today, they cover 10.8\% and 28\% of the
population live within 10 km of protected areas\footnotemark[8].
Madagascar is also characterized by a low state capacity (Hanson \&
Sigman, 2021), which makes it difficult to implement conservation and
sustainable development policies and the social measures that should
accompany them. These factors, combined with the high dependence of the
rural population on natural resources, mean that the impacts of
protected areas are potentially different from those observed in less
precarious contexts.

However, empirical studies at the national scale are almost non-existent
for Madagascar. None of the quantitative impact evaluations identified
by McKinnon et al. (2016) covered the country. One of the references
consolidated by Kandel et al. (2022) is a multi-country study that
includes Madagascar, but it is based on an estimate of an aggregate
impact at the commune level and covers only one date. It uses the 1993
census data to match the country's municipalities (Mammides, 2020),
without a before-and-after comparison, and in a context where less than
3~\% of the territory was covered by protected areas, most of which had
been created several decades earlier.

Our contribution to the literature is twofold, both empirical and
methodological. Empirically, this study provides an unprecedented
national analysis, covering 71 protected areas established between 2008
and 2021, to evaluate the socioeconomic impacts of conservation in
contexts of extreme poverty and weak governance. Methodologically, it
articulates the state of the art in econometrics, incorporating recent
developments to adapt these methods to the study of protected areas. The
procedure we propose here could be replicated in other countries,
starting with the 39 countries that have at least three geolocated DHS
surveys. This approach paves the way for a more systematic evaluation of
the impact of protected areas, taking into account the specific context
of each country.y.

\footnotetext[8]{ Calculations by the authors based on the location of the DHS survey clusters. The detailed calculation is provided as supplementary material to the study.}

\section{Research Design}\label{research-design}

\subsection{Hypothesis}\label{hypothesis}

Our first hypothesis concerns the overall impact of protected areas in
the Malagasy context. In their meta-analysis of 30 studies, Kandel et
al. (2022) report a slightly positive average impact, but highlight a
large heterogeneity of results across context. Several parameters are
likely to influence impact, as represented graphically in
\autoref{fig:logic_diagram} in the form of directed acyclic graph
(Hünermund \& Bareinboim, 2023; Imbens, 2024)

\begin{figure}[H]
    \centering %Centre l'image horizontalement sur la page
    \caption{Logic diagram of the theory of change tested in the study }
    \includegraphics[width=0.8\linewidth]{C:/Users/irian/Documents/Statistiques/PA-livelihood-impact-dhs2/Pre-analysis-plan/images/figure 1_WU_21.png} %contrôle de la taille
    \caption*{\textit{Source:Authors}} %légende
    \label{fig:logic_diagram} %référencement dans le texte
\end{figure}

The factors likely to lead to a decline in well-being seem particularly
significant in the Malagasy context, where the population is
predominantly rural and living in extreme poverty (the last assessment
was in 2012, with 80.7\% of the population below the \$2.15 a day
threshold at 2017 PPP). Six studies conducted in Madagascar between 1995
and 2006 estimated the opportunity cost of losing access to protected
areas (slash-and-burn agriculture, hunting, gathering, timber, etc.) at
between USD 39 and 177 per household per year (Neudert et al., 2017).
Golden et al. (2014) estimated that income from hunting accounted for
57~\% of household cash income in areas adjacent to the Makira and
Masoala protected areas. Another survey of people living near Makira
estimated the value of pharmaceutical use at USD 30-44 per year per
household, based on the subsidized price of equivalent treatments in the
malagasy market (Golden et al., 2012).

Several factors that could help improve livelihoods through conservation
appear to be fragile in Madagascar, starting with tourism. Naidoo et
al.~(2019) aggregate data from DHS surveys conducted between 2001 and
2011 in 34 developing countries. Their study is based on matching
households near and far from protected areas, but with no pre-post
conservation comparison. They highlight positive impacts, but only for a
subset of protected areas `with documented tourism'\footnotemark[9].
According to their study, households living near the protected areas
`with tourism'~are 17\% wealthier and 16\% less likely to be poor than
similar households living far from these areas.

\footnotetext[9]{The source used to consider that va PA has ‘documented tourism’ is not reported in Naidoo et al 2019.}

However, tourism in Madagascar's protected areas remains low. According
to data from Madagascar National Parks (MNP), only 7 protected areas
recorded more than 10,000 visitors in 2023 (with a maximum of 30,744 in
Isalo), which is low compared to the average of 356,405 visitors per
year and per PA recorded in 929 PAs worldwide in the global study by
Chung et al. (2018). When new protected areas are created in Madagascar,
compensation mechanisms for local populations remain rare, ineffective
and insufficient (Bertrand et al., 2012; Rivière, 2017). The most
in-depth study on this subject, conducted by Poudyal et al. (2018) with
support from the World Bank, focuses on the Ankeniheny Zahamena Corridor
(CAZ), created in 2015 to connect several existing protected areas. Five
study sites were selected: Two adjacent to the new CAZ protected areas
(one eligible for compensation, the other not), two adjacent to
long-established protected areas, and one far from the forest boundary.
The median cost of the conservation restriction is estimated at USD
2,375 per household per year, representing 27\% to 84\% of the average
annual income. The amounts set aside to compensate beneficiary
households were assessed to be insufficient relative to the losses
incurred, and 50\% of households eligible for compensation received
nothing (Poudyal et al., 2016, 2018).

\begin{itemize}
\tightlist
\item
  \textbf{Hypothesis 1:} Protected areas in Madagascar, by limiting
  access to natural resources, have negative impacts on the well-being
  of nearby households that often exceed the benefits of compensation
  and ecosystem services, with more adverse impacts than in other
  countries.
\end{itemize}

The impact mechanisms represented in \autoref{fig:logic_diagram} are
likely to affect households differently depending on their prior
characteristics. Compensation measures are generally implemented in the
form of projects to promote income-generating activities (agriculture,
livestock, handicrafts) in surrounding communities (Poudyal et al.,
2018). In the context of such development projects, individuals known as
`development brokers' frequently emerge as intermediaries between local
communities and implementing organizations. By mobilizing their social
networks and specific skills, these brokers manage to capture a
disproportionate share of the benefits of interventions, whether in form
of income or access to exclusive opportunities. This dynamic can
reinforce pre-existing inequalities within communities, limiting the
access of the most vulnerable households to the expected benefits of
compensation programs.

Although tourism development is often presented as an opportunity for
economic growth, it also tends to exacerbate socioeconomic inequalities,
particularly in developing countries. Adeniyi et al. (2024) show that in
Southern Africa, tourism can initially exacerbate inequalities by
concentrating benefits in the most attractive regions, while leaving
marginalized communities out of the economic benefits. According to
Ghosh and Mitra (2021) the relationship between tourism and inequality
follows an inverted Kuznets curve dynamic in developing countries, when
tourism remains moderate, its growth reduces binequalities, but when
tourism becomes massive, further expansion worsens inequalities.
Finally, Xuanming et al. (2024) point out that while tourism helps to
improve certain socioeconomic indicators, it can also generate
inflationary pressures and strain local resources, particularly
affecting the most vulnerable households.

\textbf{Hypothesis 2}: Protected areas exacerbate economic inequalities
among neighboring communities by creating opportunities that mainly
benefit individuals with a higher educational level or a dominant
position in the community, allowing them access to rents and jobs
related to tourism and associated activities.

Protected areas IUCN status is frequently used to explain differences in
effectiveness between them. For example, Naidoo et al. (2019) show that
multiple-use protected areas (statuses V and VI) tend to have more
beneficial effects than strict areas (statuses I to IV), partly due to
greater flexibility in integrating local needs. Beyond status alone,
governance plays a central role. Eklund et al. (2017) highlight the
importance of transparent and inclusive structures to maximize the
positive effects of protected areas on conservation and social justice.
Similarly, Eklund et al. (2019) call for management approaches to be
adapted to local contexts, with greater involvement of communities in
decision-making processes, to better reconcile conservation and
development objectives.

This diversity is particularly evident in Madagascar. Although governed
by similar formal statutes, protected areas follow different paths
depending on the local context and the way in which they are
implemented. Froger and Méral (2009) show that the early initiatives of
shared governance, gradually introduced with in-depth mediation efforts,
achieved encouraging results by strengthening local community support.
However, from the 2000s onward, the accelerated deployment of management
transfers, driven by quantitative targets, often led to hasty and less
contextually adapted implementations, undermining the effectiveness of
these mechanisms. These experiences demonstrate that, beyond the
protected area status, their establishment period, management approach,
and level of community participation significantly influence their
socio-economic impacts.

\begin{itemize}
\tightlist
\item
  \textbf{Hypothesis 3}: The impacts of protected areas on well-being
  and inequalities are heterogeneous, and some protected areas with good
  levels of local community participation manage to generate greater
  benefits and distribute them more equitably.
\end{itemize}

\subsection{Basic methodological framework / identification
strategy}\label{basic-methodological-framework-identification-strategy}

Our evaluative approach is based on a counterfactual measure that
estimates the causal effect of the treatment, in this case the presence
of protected areas. The counterfactual measure corresponds to a
hypothetical scenario describing what would have happened if the
intervention under study had not taken place. This approach is based on
a comparison between a treatment group (a protected areas) and a control
group (unprotected areas with characteristics very similar to those of
the protected areas). The study thus fits within the framework of
Rubin's causal model Rubin (1974), according to which there are several
hypothetical outcomes depending on exposure to the treatment. To ensure
comparability between groups, matching techniques are used to assign
each unit in the treatment group to a unit with the same observable
characteristics in the control group. Using matching increases the
credibility of research results and reduces endogeneity problems (Ma et
al., 2020).

We subsequently use the difference-in-difference method to estimate the
causal effect attributable to the creation of protected areas. This
method allows us to compare the observed changes in the treatment group
and control group, while compensating for pre-existing differences
between these two groups. By comparing the differences in local
households livelihood before and after the creation of the protected
areas, we isolate the specific effect of the creation of these protected
areas. Matching and difference-in-difference methods are often used
together to reduce selection bias. Several studies have used this
combination of methods to evaluate the impact of conservation on land
use and livelihoods (Ma et al., 2020; Schleicher et al., 2020a).

\subsection{Intervention}\label{intervention}

This study evaluates the impact of terrestrial protected areas creation
on rural household well-being between 2008 and 2021. These time frames
were chosen on the basis of the availability of geolocalised data on
household living conditions and coincide with a period of strong
expansion of protected areas in the country, as shown in

\begin{figure}[H]
    \centering %Centre l'image horizontalement sur la page
    \caption{Evolution of protected areas in Madagascar and study period}
    \includegraphics[width=0.8\linewidth]{C:/Users/irian/Documents/Statistiques/PA-livelihood-impact-dhs2/Pre-analysis-plan/images/Figure 2_WU_21.png}  %contrôle de la taille
     \caption*{\textit{Source: Calculations by authors based on data from the Service de la Gouvernance des Aires Protégées (SGAP) of the Ministère de l’environnement et Développement Durable (MEDD)}} %Source de la figure
    \label{fig:evol_diagram} %référencement dans le texte
\end{figure}

Protected areas in Madagascar were first created in 1927 under the
French colonial administration and, until the early 2000s, were mainly
characterized by strict conservation (IUCN statuses I, II, and IV). At
the fifth IUCN Parks Summit in Durban in 2003, the Malagasy government
committed to a tripling of protected areas. The declaration led to a
wave of protected area creation, with 28 provisional creation decrees
published between April 2006 and December 2007, and a global decree
bringing the number of new protected areas to 97 in 2008. These decrees
did not designate a manager, leaving it to the Ministry of the
Environment to appoint one. This seems to have been the general
practice, and managers were in place in the majority of the newly
protected areas in the following years. However, it was not until 2015
that a final decree ratified these creations. Some of these protected
areas were the subject of early management transfer decrees, between
2011 and 2015. Some uncertainty remains about the exact date of these
early transfers, and the fact that our study period covers a wider
interval (2008-2021) compensates for this inaccuracy in the start date
of treatment, i.e actual conservation.

\autoref{tbl-tab:protected_areas} presents the distribution of protected
areas ( in number and area) according to their period of designation by
decree, taking 2008 as the reference year. In the treatment period (2008
to 2021), 71 protected areas were created covering 47,282 km².

\begin{table}

\caption{\label{tbl-tab:protected_areas}Number and area of protected areas by enactment period}

\centering{

[H]
    \centering
    \captionsetup{font=small, labelfont=bf} % Amélioration de la légende
    \sisetup{group-separator = {\,}} % Séparateur des milliers
    
    %Réinitialisation du compteur des tableaux 
    \setcounter{table}{0} \renewcommand{\thetable}{\arabic{table}}
    
    \begin{tabular}{lccc|ccc}
        \toprule
        \multirow{2}{*}{\textbf{Enactment period by decree}} & \multicolumn{3}{c|}{\textbf{Protected Areas (Number)}} & \multicolumn{3}{c}{\textbf{Area (km²)}} \\
        & Terrestrial & Non-terrestrial & Total & Terrestrial & Non-terrestrial & Total \\
        \midrule
        Starting in 2008 & 71 & 9 & 80 & 47,482 & 6,815 & 54,298 \\
        Before 2008 & 43 & 0 & 43 & 21,801 & 0 & 21,801 \\
        \midrule
        Total & 114 & 9 & 123 & 69,283 & 6,815 & 76,099 \\
        \bottomrule
    \end{tabular}
    
    
    
    
    
     \caption*{\textit{Source: Calculations by authors based on data from the Service de la Gouvernance des Aires Protégées (SGAP) of the Ministère de l’environnement et Développement Durable (MEDD)}}
     
     

}

\end{table}%

\newpage

In the context of our study, the population considered as treated
encompasses households living in rural areas within 10 km of a protected
area created between 2008 and 2021, according to the GPS coordinates
provided in the Demographic Health Surveys (DHS) data. These GPS
coordinates correspond to the centroids of the enumeration areas
surveyed. To protect respondent confidentiality, these coordinates are
first randomly shifted using the following procedure: An offset angle
between 0 and 360 is randomly drawn, then an offser distance is randomly
drawn, between 0 and 2 km in urban areas and between 0 and 5 km in rural
areas. For 1\% of rural clusters, the distance drawn is between 0 and 10
km (Skiles et al., 2013). The 10 km threshold used in our analysis
corresponds to the distance most commonly used in similar impact
evaluations (see references in Appendix A). Households in the control
group are those living in a rural area more than 10 km away from a
protected area created between 2008 and 2021, and they exhibit very
similar characteristics or share significant traits with households in
the treatment group. To check the robustness of the analysis, we also
test other distances (5 km and 15 km).

We decided to exclude rural populations living within 10 km of protected
areas created before 2008, as they are considered treated before the
study period. Urban areas were also excluded from the study because the
mechanisms by which protected areas could affect the living conditions
of urban populations are likely very different from those rural
populations, which are mainly farmers. Due to the marked differences in
living standards between rural and urban areas, including urban
households would also risk unnecessarily increasing the variance of our
estimates. Furthermore, according to the 2018 census, urban population
accounts for 24\% of the Malagasy population (INSTAT, 2020a), and this
proportion is likely even lower among population living near protected
areas.

In some cases, the 10 km zone around a newly created protected area may
overlap with the 10 km zone around an existing older protected areas. In
such a case, the treatment date taken into account is the date of the
first creation in chronological order. If the older protected area was
created before 2008, the overlapping zone will be excluded from the
analysis.

\subsection{Sample and statistical
power}\label{sample-and-statistical-power}

\subsubsection{Sample}\label{sample}

The unit of analysis in our study is the household. This is indeed a
level at which a significant portion of individual resources are pooled,
and it is at this level that data on living standards -- the outcome
variable of our study -- are available in national surveys (Deaton,
1997).

The data on household living conditions used for this study comes from
surveys conducted by the `\emph{Institut National de la Statistique de
Madagascar'} (INSTAT) as part of the Demographic Health Surveys (DHS)
programs. The DHS surveys rely on two-stage stratified sampling. The
population of interest was divided into 23 study areas corresponding to
the 22 regions of Madagascar, the capital Antananarivo (considered
separately) and the Analamanga region without the capital (to isolate
the effect of the capital on the regional results). Except for capital,
two strata were created in each study area: The urban stratum and the
rural stratum. A total of 45 sampling strata were created. At the first
stage, enumeration areas (referred to as `clusters' hereafter) were
randomly selected within each area with a probability proportional to
the population of the cluster according to the most recent census. A
complete enumeration of households was conducted in each selected
cluster, with 657 clusters in 2021, 600 in 2008 and 270 in 1997. At the
second stage, a sample of households was randomly drawn from within
these clusters to be invited to participate in the survey. This method
helps to reduce survey costs while ensuring the representativeness of
the results.

Our study will be based on samples from the DHS surveys, which
constitute repeated cross-sections over what we will call a `treatment'
period of 13 years (2008-2021). To validate the comparability hypothesis
between our treatment and control groups during the treatment period, we
will analyze their trends over a `pre-treatment' period corresponding to
the preceding 11 years (1997-2008).

\subsubsection{Statistical power}\label{statistical-power}

The calculation of statistical power aims to estimate the probability
that the study detect an effect when it exists. The expected power
threshold is commonly set at 0.8, which means that the study has an
80~\% chance of rejecting the null hypothesis when an effect is present.
Furthermore, the significance threshold is set at 0.05, indicating a
5~\% risk of incorrectly rejecting the null hypothesis when it is true.

The minimal detectable effect represents the smallest difference that
the study can detect between the treatment and control groups, taking
into account the parameters chosen for power and the specification
threshold. To measure this effect, we use the standardized effect size,
also known as Cohen's , defined as follows:

\[
d = \frac{\bar{X}_1 - \bar{X}_2}{s}\]

where \(X_1\) et \(X_2\) are the average of the treatment and control
groups, and is the combined standard deviation of the two groups .

We use the average and standard deviation values from the treatment
group for both groups as the matching process has not yet been
performed. We assume that the control group, once matched, will be
similar to the treatment group in terms of variability and sample size.

Prior the power calculation, we segmented the wealth index into
percentiles rather than quintiles to better capture the continuity of
the variable and maximize the sensitivity of the econometric analysis.
The approach, already used by some researchers (Staveteig \& Mallick,
2014), preserves greater granularity in the variable while improving the
precision of the estimates. By increasing the number of classes, we will
reduce the risk of bias associated with artificial discontinuities and
improve the ability of our models to capture the marginal effects of
conservation policies on living standards.

With these characteristics, we calculate that the minimal detectable
effect is 7.5, indicating that the study is able to detect a minimum
change of 7.5 percentiles in the wealth index between the groups.

\begin{table}[h]
    \centering
    \captionsetup{font=small, labelfont=bf} % Amélioration de la légende
    \sisetup{group-separator = {\,}} % Séparateur des milliers

    \begin{tabular}{lc}
        \toprule
        \textbf{Metric} & \textbf{Value} \\
        \midrule
        Intra-Cluster Correlation & 0.4783 \\
        Adjusted Standard Deviation & 28.8680 \\
        Cohen’s d & 0.2582 \\
        Minimal Detectable State in Outcome Variable Unit & 7.4548 \\
        \bottomrule
    \end{tabular}

    \caption{Statistical power of rural household percentile wealth index study}
    
    
    \caption*{\textit{Source: Authors’ calculations from DHS 2021 data}}
    
\end{table}

\section{Data}\label{data}

\subsection{Data collection and
processing}\label{data-collection-and-processing}

Given the nature of the intervention to be evaluated (long-term
perspective, large scale, complex and politically sensitive) and the
availability of the necessary information at the national level,
experimentation does not seem relevant or possible, and our study is
observational in nature. It uses secondary data on the socioeconomic
conditions of households, their geographical environment and their
location in relation to protected areas.

\subsubsection{Data on household socioeconomic
conditions}\label{data-on-household-socioeconomic-conditions}

The DHS data used covers a wide range of topics, including demographic
characteristics, living conditions, health, education, sanitation, and
hygiene. DHS surveys were performed in Madagascar in 1997, 2008 and
2021. The DHS program also accompanied INSTAT surveys in 2011, 2013 and
2016, entitled Malaria Indicator Surveys (MIS). These surveys, which
focus specifically on malaria related issues, do not include certain
health or demographic questions that are included in the DHS surveys.
However, all the variables relating to household living conditions that
we will use in our study from the DHS surveys are also present in the
MIS surveys.

The MIS data can therefore be mobilized for this study. However, adding
comparison periods beyond a single ``before'' and ``after'' period has
the drawback of making the analysis more complex and requiring methods
that have a fragile methodological consensus. Since 2020, the state of
the art in difference-in-difference methods has indeed been disrupted by
the realization that classical two-way fixed effects methods can produce
spurious results when the effects of an intervention are heterogeneous
and evaluated over multiple periods. Since then, a dozen alternative
approaches have been developed, but econometricians continue to debate
their reliability (Roth et al., 2023). To avoid undermining the
credibility of our results, we will initially limit ourselves to two
study periods (DHS 2008 and 2021). Only if this approach proves to lack
sufficient statistical power will we mobilize additional data.

\subsubsection{Data on the household geophysical
environment}\label{data-on-the-household-geophysical-environment}

Household geophysical environment data will be obtained using the R
package mapme.biodiversity (Görgen \& Bhandari, 2022). This package
automates the retrieval of large raster-format data and their processing
to produce a series of indicators applied to user-defined polygons for
specified periods. It has the advantage of providing a list of the most
recognized sources in the literature, retrieving them automatically and
applying the state of the art to calculate spatio-temporal indicators
from these sources. It can handle large datasets using efficient
routines and parallel computing.

\subsubsection{Data on protected areas}\label{data-on-protected-areas}

Most studies of protected areas in the literature use the World Database
on Protected Areas (WDPA), which is generally considered to be the most
comprehensive register of marine and terrestrial protected areas
(Bingham et al., 2019). However, in Madagascar, where we triangulated
data with other sources (data from the official Malagasy body overseeing
protected areas, official decrees, and documentation from protected area
managers), we find that WDPA data are inaccurate for about one-third of
protected areas. This includes inaccurate classifications of IUCN
status, incorrect creation dates, and delineations that include buffer
zones extending beyond the boundaries of protected areas. Such lack of
reliability could undermine the credibility of the results if we use
WDPA data. Thus, a study of the degradation of the effectiveness of
protected areas during COVID-19 pandemic ((Eklund et al., 2022)), which
used WDPA data, was heavily criticized by (Andrianambinina et al., 2023)
as relying on sources with errors that compromised the study's
conclusions.

The authoritative Malagasy body that administers data on the boundaries
of Madagascar's protected areas is a department of the Ministry of the
Environment and Sustainable Development called the \emph{Service de la
Gouvernance des Aires Protégées} (SGAP). These are the data we want to
prioritize for our study, and we obtained the perm permission for this
use.

\section{Analysis}\label{analysis}

\subsection{Variables used in the
analysis}\label{variables-used-in-the-analysis}

The data on the socioeconomic conditions of households and their
geophysical environment described above are used to construct the
variables in the impact evaluation model.

\subsubsection{Outcome variables}\label{outcome-variables}

The outcome variable measures the impact of the intervention. In this
analysis, two outcome variables are considered: household living
standards (primary variable) and inequalities in living standards at the
level of the surveyed localities (secondary variable).

\begin{itemize}
\tightlist
\item
  \textbf{Main outcome variable: Household living standards}
\end{itemize}

The first outcome variable, household living standard, is estimated from
the wealth index, calculated specifically for rural areas (variable
coded hv270a in the DHS data). The wealth index is defined in the DHS
data catalogue as : 'A \emph{composite measure of a household's
cumulative living standard. The wealth index is calculated using
easy-to-collect data on a household's ownership of selected assets, such
as televisions and bicycles; materials used for housing construction;
and types of water access and sanitation facilities. Generated with a
statistical procedure known as principal components analysis, the wealth
index places individual households on a continuous scale of relative
wealth. DHS separates all interviewed households into five wealth
quintiles to compare the influence of wealth on various population,
health and nutrition indicators. As a response to criticism that a
single wealth index is too urban in its construction and not able to
distinguish the poorest of the poor from other poor households, this
variable provides an urban- and rural-specific wealth index''}
(Program/ICF, 2018)\emph{. As described above, we will translate this
wealth index into an integer between 1 and 100, corresponding to the
household's wealth percentile relative to the distribution of the whole
sample.}

According to the DHS variable catalog by survey generation, the variable
hv270a is available for the 2021 DHS survey, but not for previous years.
However, the procedure for calculating the wealth index is
well-documented (Rustein \& Kiersten, 2004), and the DHS program
provides the SPSS codes for overall calculations encompassing both rural
and urban areas (variable hv270), as well as the weights associated with
each asset category for the variable hv270\footnotemark[10]. These
elements will allow us to reproduce in R and validate the calculation of
the variable hv270 (wealth index) in 2008 and 1997, and then extrapolate
the method to obtain the variable hv270a (rural wealth index) for these
years.

\footnotetext[10]{Available     at  \href{https://www.dhsprogram.com/topics/wealth-index/Wealth-Index-Construction.cfm/}{https://www.dhsprogram.com/topics/wealth-index/Wealth-Index-Construction.cfm/}, consulted on 30/08/2024}

\begin{itemize}
\tightlist
\item
  \textbf{Secondary outcome variable: inequality of household living
  standards}
\end{itemize}

In addition to the evaluation impact of protected areas on household
living standards, we will seek to understand their influence on
socioeconomic inequalities within the affected populations. To do this,
we propose to use a standardized Z-Score of the wealth index, allowing
for the comparison of the relative distribution of wealth around the
mean within the study population, at the level of each survey cluster.

The Z-Score \(Z_i\) for each household is calculated from the wealth
index using the following formula:

\[
Z_i = \frac{W_i - \mu_W}{\sigma_W}
\]

where \(μ_W\) represents the average wealth index for all households
surveyed in each rural cluster and \(σ_W\) is the standard deviation of
households in the cluster.

\subsubsection{Matching variables}\label{matching-variables}

Matching covariates are used to select units not exposed to conservation
(control groups) that are comparable to the exposed units (treatment
group). Appropriate covariates for the matching process are variables
likely to influence both the probability of treatment (whether a
protected area has been created near the household) and the outcome
(household living standard and inequalities among households). The
literature shows that protected areas tend to be created in less dense,
less accessible, higher and steeper regions (Joppa and Pfaff 2010).
These variables may also affect living standards~: areas that are more
dense, flat, low-lying and accessible (in terms of travel time and
geography) tend to be wealthier (Gallup, Sachs, and Mellinger 1999). We
will use the following list of matching variables~:

\begin{itemize}
\item
  \ul{Forest cover rates in 2000} correspond to the percentage coverage
  by vegetation with a height of 5 meters or more (Hansen et al., 2013).
  This variable is provided by the Global Forest Change dataset, which
  indicates for each pixel of one arc-second (approximately 30 meters at
  the equator) an estimate of forest cover in 2000 (Hansen et al.,
  2013). In Madagascar, as in other countries, protected areas have been
  preferentially created by targeting forest zones(Carvalho et al.,
  2020; Wilson et al., 2006). Globally, Naidoo et al. (2019) indicates
  that forest cover in an area is negatively correlated with the living
  standards of its inhabitants.
\item
  \ul{Slope and elevation} are calculated using a digital terrain model
  from NASA-SRTM (Shuttle Radar Topography data), which provides an
  elevation estimate for arc-second pixels (NASA JPL, 2020). Slope is
  measured as a percentage for each plot, while elevation is measured in
  meters. These topological variables influence the location of
  protected areas (Joppa \& Pfaff, 2010), as well as the agricultural
  potential of an area and therefore the living standards of rural
  populations (Canavire-Bacarreza \& Hanauer, 2013).
\item
  \ul{Population density in 2000} corresponds to the estimated number of
  inhabitants per km² based on Worldpop data. Worldpop data provide
  estimates of population density for the year 2000 at a spatial
  resolution of about 1 km. They use modeling techniques and combines
  census data with various geospatial datasets (WorldPop \& CIESIN,
  2018).
\item
  \ul{Accessibility in 2000} corresponds to the estimated travel time
  for households to the nearest cities, measured in minutes~; These
  accessibility data to cities are compiled by the Joint Research Center
  (JRC), with 2000 as the reference year (Uchida \& Nelson, 2011).
  Accessibility to cities determines the ability to benefit from the
  services, products and opportunities they offer and is therefore a key
  factor in determining living standards in rural areas (INSTAT, 2020b;
  Weiss et al., 2018).
\end{itemize}

Each of these variables will be calculated for a circle of 10 km radius
around the GPS coordinates of the survey cluster.

\subsubsection{Control variables}\label{control-variables}

Matching variables listed above will be integrated as control variables
in the difference-in-difference regression stage to increase the
robustness of the model by minimizing the bias that would arise from
residual imbalances. Other variables, such as the age and sex of the
head of the household or rainfall, do not introduce bias as they do not
influence treatment assignment (the location of protected areas), but
they increase the variance of the results. They are added as control
variables to reduce the unexplained variability of the model and thus
more accurately estimates the treatment effect.

The variables on age and sex of the head of household are taken from the
DHS surveys. These characteristics determine the economic opportunities
and resource management capacity of households (Lo Bue et al., 2022).
Sex of the head of household is recoded as v151 and v152 \emph{in the
DHS survey.}

We also integrate the Standardized Precipitation Evatranspiration Index
(SPEI) (Vicente-Serrano et al., 2010) for the year preceding the survey.
This index is calculated based on a long-term reference (1981-2010) to
quantify excess or deficit of rainfall. We use the SPEI for the year
prior to the survey. The SPEI is calculated from monthly precipitation
and minimum and maximum temperature data from worldclim, using an
improved version of the Hargreaves method defined by (Droogers \& Allen,
2002), as imploemented in the SPEI R package (Vicente-Serrano et al.,
2010). This variable will be calculated for a circle with a radius of 10
km around the GPS coordinates of the cluster.

We chose not to include migration among the controls, as this variable
is likely to be influenced by the direct consequences of protected areas
(reduction in land use, jobs related to conservation, etc.), but also by
the evolution of living standards in areas surrounding the protected
areas (see \autoref{fig:logic_diagram}). This position in the logical
diagram presented in \autoref{fig:logic_diagram} makes migration a
`collider', which would introduce an artificial bias if it were taken as
a control variable (Hünermund \& Bareinboim, 2023; Imbens, 2020).

\subsubsection{Weights}\label{weights}

In the DHS data, each observation is associated with a sampling weight,
reflecting the inverse of its probability of selection in the sample. We
will take these weights into account to maintain the representativeness
of our estimates relative to the national population. In addition, when
we use the genetic matching algorithm (see below), observations are
assigned an additional weight aimed at improving comparability between
the treatment and control groups. To obtain a final weighting that
accounts for both representativeness (sampling weight) and comparability
(matching weight), we multiply these two types of weights. Thus, each
observation is assigned a final weight that incorporates both
dimensions.

\subsection{Statistical methods and
models}\label{statistical-methods-and-models}

The statistical methods used in this analysis combine matching and
difference-in-difference (DID) to estimate the causal effect of
protected areas on household livelihoods.

\subsubsection{Matching}\label{matching}

Matching maximizes comparability between treatment and control groups in
terms of observable characteristics (Ho et al., 2007). This method
increases the similarity between households living in or within 10 km of
protected areas (treatment group) and households not affected by
protected areas (control group) (Desbureaux, 2021; Schleicher et al.,
2020b).

We will perform the matching using the genetic matching method proposed
by the GenMtach() function in the R package MatchIt. This variant of
nearest neighbor matching combines all variables into a single measure
of distance known as Mahalanobis distance. This distance measures the
difference between units in the matched groups to quantify the
similarity between the two groups of observations, while accounting for
correlations between the covariates and their covariances ((Diamond \&
Sekhon, 2013), p.~935.

Rosenbaum (1983) also suggests using a distance measure in combination
with allowed maximum distances between matched units (called `calipers')
to avoid creating pairs with excessively large differences. In practice,
the caliper is commonly set at an interval of 0.25 standard deviations
of the Mahalanobis distance, which corresponds to the most common
recommendation in the literature.

Matching models rely on the assumption that the distributions of the
covariates are similar between the treatment and control groups, making
the evaluation of the covariate balance crucial. The validity of the
estimates depends directly on this balance, highlighting the importance
of conducting tests to measure it.

We will conduct a two-step balance test: before matching to determine
the extent to which the raw data are initially unbalanced and to check
whether balance has been achieved between the covariates after matching.
This balance is checked using the Standardized Mean Difference (SMD),
which measures the difference between the means of the covariates in the
treatment and control groups in order to compare the relative balance of
variables measured in different units. The difference is then divided by
a standardization factor to put the difference on a common scale for all
covariates. The closer the SMD is to 0, the better the balance between
the groups for the covariate in question, and although there is no
consensus in the literature, an SMD greater than 0.1 can be considered
indicative of a significant imbalance (Austin, 2009). If the SMD exceeds
0.1, we will increase the caliper interval to achieve an SMD ≤ 0.1.

After matching, a visual test of the quality of the matching can be
performed using the quantile-quantile plot (Q-Q plot) for each matching
covariate. The graphical method provides an overview of the distribution
of the covariates. The plot facilitates the identification of imbalances
at different levels of the distribution. When the Q-Q plot points are
close to the diagonal, the quantiles of the two distributions are
similar, indicating good balance between the groups.

Matching is the only stage of the analysis where observations are
excluded. Descriptive statistics will be performed on this excluded
population to fully understand the analysis universe.

The result of genetic matching is a selection of matched observations
and associated weights. These weights will be combined with the survey
weights.

\subsubsection{Difference-in-difference}\label{difference-in-difference}

The difference-in-difference principle is to compare the wealth index of
control and treatment households before and after the establishment of
protected areas. Our treatment began in 2011, (the area is officially
protected and operationally managed), so we use 2008 DHS data for the
pre-treatment and 2021 DHS data for the post-treatment. This method
relies on the parallel trends assumption, meaning that, in the absence
of treatment, treatment and control groups would have experienced
similar changes over time. To verify this assumption, we include two
pre-treatment, points(DHS 1997 and DHS 2008 data) and assess trends
between these years.

The simplified equation for the difference-in-difference (Daw \&
Hatfield, 2018) is as follows~:

\[
Y_{it} = \alpha_0 + \beta_1 \text{Treatment}_i + \beta_2 \text{Post}_t + \delta D_{it} + \varepsilon_{it}
\]

With the parameters defined as:

\(Y_{i t}\): wealth index for household \(i\) in year \(t\) (1997, 2008
or 2021)

\(α_0\): Intercept, representing the baseline wealth index

\(β_1\): coefficient associated with the variable \(Post_t\)

\(Post_t\): Binary variable that takes the value of 1 for the
post-treatment year 2021 and 0 for the pre-treatment year 2008

\(δ\): coefficient of interest representing the effect of protected area
on household living standards in 2008 and 2021

\(D_{i t}\): interaction term between treatment and period that is equal
to 1 for treatment households after the treatment year and 0 for the
other households.

\(ε_{i t}\): error term

The difference-in-difference method relies on the assumption of parallel
trends. To validate this assumption, we will use as a reference, among
the rural households surveyed in 1997, those living in an area located
in or within of a protected area created between 2008 and 2021 (placebo
treatment group) and those matched to them using the method described
above (placebo control group). We will graphically represent the
evolution of the treatment and control groups in 1997, 2008 and 2021 for
visual confirmation. To confirm the validity of the parallel trends
assumption, we conduct a placebo test 1997-2008 outcomes, as well as
matching and control variables, while defining the treatment variable
for post-2008 protected areas. If the estimated effect between 1997 and
2008 is null or statistically insignificant, this supports the parallel
trends assumption.

Furthermore, if we incorporate data from MIS 1997, MIS 2011, and MIS
2013, we will apply an estimator suitable for staggered adoption and
multiple study periods accounting for potential heterogeneous treatment
effects (Borusyak et al., 2024). This only be performed if power is
insufficient with two periods of data.

Robust standard errors will be clustered at the enumeration area level,
addressing intra-cluster correlations to ensure accurate inference.
Observations with missing values for regression variables will be
excluded. If exclusions exceed 2\% of the sample, we will analyze their
characteristics of excluded observations and discuss potential
corrective measures to address bias associated with exclusion.

\subsection{Robustness and sensitivity
tests}\label{robustness-and-sensitivity-tests}

\subsubsection{Alternative distance
thresholds}\label{alternative-distance-thresholds}

We set a threshold of 10 km to define the treatment group, as is common
practice in the literature (Appendix A). To verify the robustness of our
conclusions, we also conduct additional analyses using distances of 5 km
and 15 km. At 5km, smaller sample sizes may lead to wider confidence
intervals, limiting statistical power. However, we will assess whether
the direction and magnitude of the estimated impacts remain consistent
across these thresholds. This comparison will help determine how
proximity to protected areas affects treatment effect.

\subsubsection{Multiple outcome and multiple hypothesis
testing}\label{multiple-outcome-and-multiple-hypothesis-testing}

This study evaluates three hypotheses: H1 (overall impact on living
standards), H2 (impact on inequalities) and H3 (heterogeneous impact
based on governance). H1 is our primary hypothesis. To reduce the risk
of Type I errors in secondary hypothesis H2 and H3, we will apply
Benjamini-Hochberg's (1995) false discovery rate. This correction
ensures that conclusions frawn from H2 and H3 are robust to multiple
testing concerns.

\subsubsection{Pseudo panel approach}\label{pseudo-panel-approach}

To account for unobserved factors or household-level shocks potentially
correlated with the outcome variable, we will use a pseudo-panel
approach to estimate fixed effects models at the level of household
cohorts (Deaton, 1985). This method estimates fixed effects models by
grouping individuals into exogenous, time-invariant cohorts (e.g., by
household head's year of birth, sex, and region). These grouping
variables ensure within-cohort variation does not bias results.
Following Jung et al. (2019), we will construct these cohorts as
follows: ``« \emph{In order to get consistent estimates from the
pseudo-panel approach, grouping variables need to be exogenous,
time-invariant, and available for all household in the data
(Verbeek,2008). We use household head's sex, birth year, and the region
as grouping variables, with birth year divided by quartile, variables
commonly used in the literature (e.g., Bernard, et al.~2011~; Pless and
Fell, 2017). Owing to the variation in cohort size between years, which
leads to heteroscedasticity, we weigh the observations using the inverse
of the square root of cohort size (Dargay, 2007~; Pless and Fell,
2017)~''.}

The model equation is as follows:

\[
\overline{Y}_{ct} = \theta_c + D_{ct} + \sum_{k=2015}^{2021} \beta \overline{Dist}_{ct} \mathbb{1}(t = k) + \delta \overline{X}_{ct} + \epsilon_i
\]

\({{Y}_{c t}}\): Average value of the household \(i\) wealth index
within the cohort \(c\) at the period \(t\)

\({θ}_{c}\): Fixed effect controlling for the time-invariant
cohort-specific characteristics

\({D}_{c t}\) :Time-fixed effects at cohort level

\({{Dist}_{ct}}\): Average distance between the location of households
and the boundaries of protected areas ( where\({{Dist}_{ct}} ∢ 10 km\)
within the cohort).

\({{X}_{c t}}\) : Average of the control variables (rainfall, drought)

\({ϵ}_{i}\) : Error terms

\subsubsection{Rosenbaum sensitivity
analysis}\label{rosenbaum-sensitivity-analysis}

To evaluate the robustness of the results against potential biases
arising from unobserved confounding variables, we will perform a
sensitivity analysis using Rosenbaum's ((Rosenbaum, 2002), 105‑70)
method. This analysis tests the extent to which potential unobserved
biases could alter the results ((Keele, 2010), 7). A sensitivity
parameter \(Γ\) measures the degree of deviation from random treatment
assignment. We will test a range of \(Γ\) values to assess how robust
our conclusions are to unobserved confounding. If results remain
significant for plausible values of \(Γ\), we can conclude that
unobserved bias is unlikely to explain our findings.

\subsubsection{Heterogeneous effects}\label{heterogeneous-effects}

To investigate heterogeneous effects of protected areas, we adopt the
following approaches. First, we introduce differentiating variables (sex
and age of the head of household, rainfall) to test whether the impact
of protected areas varies according to these characteristics
(e.g.~during wet periods vs dry periods). We also analyze the potential
influence of governance by distinguishing between so-called ``strict''
protected areas (IUCN categories I-IV) and ``multi-use'' areas (IUCN
categories V-VI). Furthermore, the use of a pseudo-panel (grouping
individuals into exogenous cohorts, for example by age or region) will
allow us to examine whether specific cohorts exhibit a more pronounced
impact while better controlling for unobserved factors. Next, we study
intra-community inequalities using Z index (z-score) of the wealth
index; this allows us to see if the effect of protected areas is more
concentrated on certain households rather than uniformly affecting the
entire local population. Finally, we test different distances (5 km, 10
km, and 15 km) to evaluate whether proximity to protected areas
determines the magnitude of the impact. All of these analyses contribute
to a better understanding of the factors and populations for which
conservation generates differentiated effects

\section{Interpreting results}\label{interpreting-results}

This study evaluates how the creation of protected areas affects the
living standards and socioeconomic inequalities of rural households in
Madagascar. The results aim to provide actionable guidelines for
conservation policies that balance biodiversity protection with the
needs and rights of local communities.

Hypothesis 1 assumes that, on average, protected areas reduce the living
standards of riparian households. If our matching and
difference-in-differences estimates show a decline in the wealth index
for these households, this would indicate that restricted access to
natural resources, coupled with insufficient compensation or
benefit-sharing mechanisms, has adversely affected their living
standards. This conclusion would then invite public authorities and
conservation stakeholders to put in place more inclusive mechanisms,
such as better sharing of tourism revenues, targeted cash transfers or
maintaining regulated access to certain resources, to prevent increased
vulnerability.

If the z-score results show that conservation exacerbates inequalities
(H2), this would suggest that `local elites' disproportionately benefit
from financial opportunities linked to tourism or protected area
management. In such cases, corrective measures would be necessary to
prevent benefits from being concentrated among a small group. Examples
include enforcing local employment quotas or ensuring transparent
participatory governance.

Cif the analysis does not show a decline in livelihoods, this would
indicate that some households have successfully diversified their
incomes or benefited from ecosystem services such as water access,
erosion control, soil fertility, or pollination. Such an observation
would strengthen the thesis that, in certain favorable institutional or
market contexts, investment in biodiversity can coexist with dynamic
local development. This observation would then encourage the expansion
of conservation while replicating the approaches that are most effective
in stimulating or maintaining household livelihoods. If, in addition,
the analysis demonstrates the absence of negative effects on
inequalities, or even an effect that promotes their alleviation, this
would strengthen the social and political acceptability of a rapid
expansion of protected areas.

Finally, hypothesis 3 concerns the heterogeneity of effects based on the
type of governance of protected areas. If the study finds that certain
management approaches---such as those involving local communities-- lead
to more favorable and equitable outcomes, these results would support
scaling up participatory models that have been promoted over the past
fifteen years. Conversely, if participatory governance does not appear
to play a decisive role, this would suggest prioritizing other
explanatory factors, such as donor support or local context, before
drawing broader conclusions about the social effectiveness of
conservation.

These results will contribute to the debate on the compatibility of
conservation and poverty reduction by identifying how institutional
configurations or governance models influence the vulnerability of rural
households. They will provide new insights for analyzing biodiversity
protection policies in low-income countries.

\section{Appendix}\label{appendix}

\textbf{Appendix A}

\begin{landscape} % Mode paysage
\begin{table}[ht]
    \centering
    \footnotesize % Texte compact mais lisible
    \renewcommand{\arraystretch}{1.1} % Espacement vertical réduit
    \setlength{\tabcolsep}{2pt} % Réduction de l'espace horizontal entre colonnes
    \renewcommand{\arrayrulewidth}{0.5mm} % Épaisseur des bordures

    \caption{Comparable study parameters of the impact of spatialized policies on household livelihoods}
    

    \resizebox{\textwidth}{!}{ % Ajuste le tableau pour qu'il tienne sur la page
    \begin{tabular}{|p{2.5cm}|p{4cm}|p{2.5cm}|p{3cm}|p{3cm}|p{3cm}|p{2.8cm}|p{2.8cm}|p{3cm}|p{3cm}|} 
        \hline
        \textbf{References} & \textbf{Title} & \textbf{Study coverage} & \textbf{Treatment variable} & \textbf{Outcome variable} & \textbf{Matching method} & \textbf{Matching variables} & \textbf{Control variables} & \textbf{Estimation methods} & \textbf{Comments} \\ 
        \hline
        
        Naidoo et al. (2019) & Evaluating the impacts of protected areas on human well-being & 84 developing countries & Households near protected areas & Children height-for-age, stunting, Wealth index, Probability of being poor ($<$100) & Genetic matching & Tree cover, Distance to nearest road, Year of DHS survey Elevation, Annual precipitation, Population density, Education, level male-headed household, Anthropogenic pressure index, Urban household, Child breastfed time to nearest, water source & Tree cover, Distance to nearest road, Year of DHS survey Elevation, Annual precipitation, Population density, Education, level male-headed household, Anthropogenic pressure index, Urban household, Child breastfed time to nearest, water source & Bayesian hierarchical regression & Hypothesis: Households living within 10 km of a PA close enough to be affected by PA presence, No comparison before and after treatment \\ 
        \hline
        
        Jung et al., 2019(Journal of the Association of Environmental and Resource Economists) & Evidence on Wealth-Improving Effects of Forest Concessions in Liberia & Liberia (2007-2013) & Households within 5 km of concession boundaries & Wealth score quartile (DHS) & Mahalanobis distance matching + caliper method after matching & under five, Household head sex, Household head age, livestock, Bank, Road, forest town, Other concessions & Religion, Education & Difference-in-difference(event study specification) & Mention quartiles but wealth index are normally available in quintiles  \\ 
        \hline
        
        Nowakowski et al., 2023 (nature sustainability) & Co-benefits of marine protected areas for nature and people & MesoAmerican region (2005-2018) & Proximity to marine protected areas & Fish abundance (biomass, diet), Wealth index, Stunting of children (height for age) & Mahalanobis distance matching + caliper method after matching & Fish abundance, Coral cover, Reef, Habitat, Depth human Footprint Chlorophyll-a
climate change, MPA fishing restrictions, MPA enforcement, MPA age, MPA area, Distance to coat, Natural land cover, travel time to cities, Population density, elevation, Education level, Country (HN), Distance to coast, MPA fishing restrictions & Fish abundance, Coral cover, Reef, Habitat, Depth human Footprint Chlorophyll-a climate change, MPA fishing restrictions, MPA enforcement, MPA age, MPA area, Distance to coat, Natural land cover, travel time to cities, Population density, elevation, Education level, Country (HN), Distance to coast, MPA fishing restrictions & Bayesian GLMMs & ~ \\ 
         \hline
        
        Miranda et al, 2014(Inter-American Bank) & Effects of Protected Areas on Forest Cover Change and Local Communities & Peru (2000-2005) & Protected areas & Deforestation disturbance, Capita income, capita expenditure, poverty rate
extreme poverty & Mahalanobis distance matching & Elevation, Slope, Aspect, Average precipitation, Average maximum temperature, Average mean temperature, Distance to nearest, Population center $> 10 000 hab$, Distance to nearest population center, Proportion land suitable for forestry, Water source in house, Electric lighting literacy, Primary school, Education, Employment in agriculture or forestry, Population density & Elevation, Slope, Aspect, Average precipitation, Average maximum temperature, Average mean temperature, Distance to nearest, Population center $> 10 000 hab$, Distance to nearest population center, Proportion land suitable for forestry, Water source in house, Electric lighting literacy, Primary school, Education, Employment in agriculture or forestry, Population density & ~ & ~ \\ 
        \hline
        
        Canavire-Bacarrezaet al 2019 & Estimating the Impacts of Bolivia’s Protected Areas on Poverty & Bolivia (1992-2001) & Protected areas & poverty index & Genetic matching + caliper & Distance to major city, Roadless volume, Elevation, Slope, Forest cover & Distance to major city, Roadless volume, Elevation, Slope, Forest cover & Doubly Robust Matching & ~  \\ 
        \hline
        
        Duan et al, 2017 (Elsevier) & Impacts   of protected areas on local livelihoods: Evidence of giant panda biosphere reserves in Sichuan Province, China & China (2015) & Households near PAs & Household wealth index & Matching & annual income per capita, Participatory wealth ranking, crop production, Timber forest, NTFPs gathering, Animal husbandry, Non-farm work, Age of household head (yrs), Education of household head(yrs), Adult-equivalent units, Labor quantity, Resource variables, Total cropland area, Cropland area per plot, Total forestland area, Forestland area per plot, Slope of forestland, Distance  of village center to nearest town, Number of clinics Number of roads & annual income per capita, Participatory wealth ranking, crop production, Timber forest, NTFPs gathering, Animal husbandry, Non-farm work, Age of household head (yrs), Education of household head(yrs), Adult-equivalent units, Labor quantity, Resource variables, Total cropland area, Cropland area per plot, Total forestland area, Forestland area per plot, Slope of forestland, Distance     of village center to nearest town & Ordinary    regression model & ~  \\
        \hline
        
        Clements et al, 2014(Elsevier) & Impacts    of Protected Areas on Local Livelihoods in Cambodia & Cambodia (2008) & Households near PA & BNS Score household rice harvest & Matching (with the package « nlme ») + 0,5 caliper +mixed effects models & Number families in the village in 2005, Distance to nearest all-weather road in 2005, Distance to nearest full-day market in 2005, Percentage of forest cover within 5 km of the village & Number families in the village in 2005, Distance to nearest all-weather road in 2005, Distance to nearest full-day market in 2005, Percentage of forest cover within 5 km of the village & ~ & ~ \\ 
        \hline
        
        Den Braber et al, 2018
(Conservation letters) & Impact of protected areas on poverty, extreme poverty, and inequality in Nepal & Népal (2001-2011) & Village development committees (VDCs) covering 32AP & poverty & genetic matching & levels of our poverty measures, Slope, Elevation, precipitation, VDC area, Forest cover travel time to the nearest PA Entrance, Major and minor trekking routes, Proportion of the VDC under Community, forest Management and the age of community, forestry arrangement, Population density, agricultural effort, International migration, District & levels of our poverty measures, Slope, Elevation, precipitation, VDC area, Forest cover travel time to the nearest PA Entrance, Major and minor trekking routes, Proportion of the VDC under Community, forest Management and the age of community, forestry arrangement, Population density, agricultural effort, International migration, District & OLS regression  matched binomial regression & ~ \\
        \hline
    \end{tabular}
    } % Fin du resizebox
    \caption*{\textit{Source: Authors}}
\end{table}
\end{landscape}

\newpage

\section{Administrative information}\label{administrative-information}

\subsection{Funding}\label{funding}

The study is performed in the framework of the BETSAKA project. The
BETSAKA project is cofunded by the Development Impact Lab of the German
KfW Development Bank; the \emph{Agence Française de Développement} (AFD)
and the French Research Institute for Sustainable Development (IRD).

\subsection{Institutional Review Board (Ethical
approval)}\label{institutional-review-board-ethical-approval}

The project includes the analysis of personal data collected through
surveys. In order to manage these data effectively and ethically, a data
management plan will be implemented. This plan will comply with the good
practices defined by authoritative bodies such as the French Ministry of
Research in its National Plan for Open Science and the European
Commission in the guidelines of the European Research Council (ERC).
These good practices aim to ensure that the collection, storage,
processing and sharing of data are conducted in a manner that respects
privacy, ensures confidentiality and maintains data integrity. The main
aspects of these guidelines include obtaining informed consent from
participants, the anonymity of personal data when necessary, and
establishing clear protocols for data access and sharing. All
participants in the research project will therefore be informed and
trained to respect these rules and best practices.

The implementation of the data management plan is based on the
DMP-Opidor system developed by INIST-CNRS, recommended by the
'\emph{Agence Nationale de la Recherche et l'Institut de Recherche et de
Développement' (IRD).} DMP-Opidor is a tool designed to help researchers
create, manage and share their data management plan. It provides a
structured framework that guides researchers through the various stages
of data management, ensuring compliance with legal and ethical
requirements.

\subsection{Declaration of interest}\label{declaration-of-interest}

Iriana Razafimahenina, Florent Bédécarrats, Ingrid Dallmann et
Holimalala Randriamanampisoa have declared that they have no conflicts
of interest.

\section{Acknowledgements}\label{acknowledgements}

We greatly appreciate the contribution of the researchers and
conservation officers in Madagascar through the discussions and
exchanges we were able to have.

\section*{References}\label{references}
\addcontentsline{toc}{section}{References}

\phantomsection\label{refs}
\begin{CSLReferences}{1}{0}
\bibitem[\citeproctext]{ref-adams_biodiversity_2004}
Adams, William. M., Aveling, R., Brockington, D., Dickson, B., Elliott,
J., Hutton, J., Roe, D., Vira, B., \& Wolmer, W. (2004). Biodiversity
{Conservation} and the {Eradication} of {Poverty}. \emph{Science},
\emph{306}(5699), 1146--1149.
\url{https://doi.org/10.1126/science.1097920}

\bibitem[\citeproctext]{ref-adeniyi_tourism-income_2024}
Adeniyi, O., Adekunle, W., Afolabi, J., \& Kumeka, T. T. (2024).
Tourism-income inequality {Nexus} in {Africa}: Evidence from {SADC}
countries. \emph{Current Issues in Tourism}, \emph{27}(18), 2899--2917.
\url{https://doi.org/10.1080/13683500.2023.2227377}

\bibitem[\citeproctext]{ref-andrianambinina_decrease_2023}
Andrianambinina, O. D., Rafanoharana, S. C., Rasamuel, H. A. T., Waeber,
P. O., Ganzhorn, J. U., \& Wilmé, L. (2023). Decrease of deforestation
in {Protected} {Areas} of {Madagascar} during the {Covid}-19 years.
\emph{Madagascar Conservation \& Development}, \emph{18}(1), 15--21.
\url{https://www.ajol.info/index.php/mcd/article/view/262765}

\bibitem[\citeproctext]{ref-austin_balance_2009}
Austin, P. C. (2009). Balance diagnostics for comparing the distribution
of baseline covariates between treatment groups in propensity-score
matched samples. \emph{Statistics in Medicine}, \emph{28}(25),
3083--3107. \url{https://doi.org/10.1002/sim.3697}

\bibitem[\citeproctext]{ref-bertrand_contre_2012}
Bertrand, A., Serpantié, G., Randrianarivelo, G., Montagne, P.,
Toillier, A., Karpe, P., Andriambolanoro, D., \& Derycke, M. (2012).
Contre un retour aux barrières~: Quelle place pour la gestion
communautaire dans les nouvelles aires protégées malgaches~? \emph{Les
Cahiers d'Outre-Mer}, \emph{257}(1), 85--123.
\url{https://doi.org/10.4000/com.6493}

\bibitem[\citeproctext]{ref-bingham_sixty_2019}
Bingham, H. C., Juffe Bignoli, D., Lewis, E., MacSharry, B., Burgess, N.
D., Visconti, P., Deguignet, M., Misrachi, M., Walpole, M., \& Stewart,
J. L. (2019). Sixty years of tracking conservation progress using the
{World} {Database} on {Protected} {Areas}. \emph{Nature Ecology \&
Evolution}, \emph{3}(5), 737--743.
\url{https://idp.nature.com/authorize/casa?redirect_uri=https://www.nature.com/articles/s41559-019-0869-3&casa_token=Zmluq7JmhxUAAAAA:Id6cAJrCDpmtydpiga8LlmkB5wb9a1O_bwCGwCeA1gz314-F6VT5ekCOeVMJo_Zrpbh_uuoXf0fwD3Txjg}

\bibitem[\citeproctext]{ref-borusyak_revisiting_2024}
Borusyak, K., Jaravel, X., \& Spiess, J. (2024). Revisiting
{Event}-{Study} {Designs}: {Robust} and {Efficient} {Estimation}.
\emph{The Review of Economic Studies}, rdae007.
\url{https://doi.org/10.1093/restud/rdae007}

\bibitem[\citeproctext]{ref-canavire-bacarreza_estimating_2013}
Canavire-Bacarreza, G., \& Hanauer, M. M. (2013). Estimating the impacts
of {Bolivia}'s protected areas on poverty. \emph{World Development},
\emph{41}, 265--285.
\url{https://www.sciencedirect.com/science/article/pii/S0305750X12001702}

\bibitem[\citeproctext]{ref-carvalho_methods_2020}
Carvalho, F., Brown, K. A., Gordon, A. D., Yesuf, G. U., Raherilalao, M.
J., Raselimanana, A. P., Soarimalala, V., \& Goodman, S. M. (2020).
Methods for prioritizing protected areas using individual and aggregate
rankings. \emph{Environmental Conservation}, \emph{47}(2), 113--122.
\url{https://doi.org/10.1017/S0376892920000090}

\bibitem[\citeproctext]{ref-chung_global_2018}
Chung, M. G., Dietz, T., \& Liu, J. (2018). Global relationships between
biodiversity and nature-based tourism in protected areas.
\emph{Ecosystem Services}, \emph{34}, 11--23.
\url{https://doi.org/10.1016/j.ecoser.2018.09.004}

\bibitem[\citeproctext]{ref-conceicao_breaking_2024}
Conceição, P. (Ed.). (2024). \emph{Breaking the gridlock: {Reimagining}
cooperation in a {Polarized} world}. UNDP.

\bibitem[\citeproctext]{ref-daw_matching_2018}
Daw, J. R., \& Hatfield, L. A. (2018). Matching and {Regression} to the
{Mean} in {Difference}-in-{Differences} {Analysis}. \emph{Health
Services Research}, \emph{53}(6), 4138--4156.
\url{https://doi.org/10.1111/1475-6773.12993}

\bibitem[\citeproctext]{ref-deaton_panel_1985}
Deaton, A. (1985). Panel data from time series of cross-sections.
\emph{Journal of Econometrics}, \emph{30}(1-2), 109--126.
\url{https://doi.org/10.1016/0304-4076(85)90134-4}

\bibitem[\citeproctext]{ref-deaton_analysis_1997}
Deaton, A. (1997). \emph{The {Analysis} of {Household} {Surveys}: {A}
{Microeconometric} {Approach} to {Development} {Policy}}. World Bank
Publications.

\bibitem[\citeproctext]{ref-desbureaux2021}
Desbureaux, S. (2021). Subjective modeling choices and the robustness of
impact evaluations in conservation science. \emph{Conservation Biology},
\emph{35}(5), 16151626. \url{https://doi.org/10.1111/cobi.13728}

\bibitem[\citeproctext]{ref-diamond_genetic_2013}
Diamond, A., \& Sekhon, J. S. (2013). Genetic {Matching} for
{Estimating} {Causal} {Effects}: {A} {General} {Multivariate} {Matching}
{Method} for {Achieving} {Balance} in {Observational} {Studies}.
\emph{Review of Economics and Statistics}, \emph{95}(3), 932--945.
\url{https://doi.org/10.1162/REST_a_00318}

\bibitem[\citeproctext]{ref-droogers_estimating_2002}
Droogers, P., \& Allen, R. G. (2002). Estimating {Reference}
{Evapotranspiration} {Under} {Inaccurate} {Data} {Conditions}.
\emph{Irrigation and Drainage Systems}, \emph{16}(1), 33--45.
\url{https://doi.org/10.1023/A:1015508322413}

\bibitem[\citeproctext]{ref-eklund_quality_2017}
Eklund, J., \& Cabeza, M. (2017). Quality of governance and
effectiveness of protected areas: Crucial concepts for conservation
planning. \emph{Annals of the New York Academy of Sciences},
\emph{1399}(1), 27--41. \url{https://doi.org/10.1111/nyas.13284}

\bibitem[\citeproctext]{ref-eklund_what_2019}
Eklund, J., Coad, L., Geldmann, J., \& Cabeza, M. (2019). What
constitutes a useful measure of protected area effectiveness? {A} case
study of management inputs and protected area impacts in {Madagascar}.
\emph{Conservat Sci and Prac}, \emph{1}(10), e107.
\url{https://doi.org/10.1111/csp2.107}

\bibitem[\citeproctext]{ref-eklund_elevated_2022}
Eklund, J., Jones, J. P., Räsänen, M., Geldmann, J., Jokinen, A.-P.,
Pellegrini, A., Rakotobe, D., Rakotonarivo, O. S., Toivonen, T., \&
Balmford, A. (2022). Elevated fires during {COVID}-19 lockdown and the
vulnerability of protected areas. \emph{Nature Sustainability},
\emph{5}(7), 603--609.

\bibitem[\citeproctext]{ref-froger_temps_2009}
Froger, G., \& Méral, P. (2009). Les temps de la politique
environnementale à {Madagascar} : Entre continuité et bifurcations. In
\emph{Diversité des politiques de développement durable : Temporalités
et durabilités en conflit à {Madagascar}, au {Mali} et au {Mexique}}
(pp. 45--67). Karthala ; GEMDEV.
\url{https://www.documentation.ird.fr/hor/fdi:010048701}

\bibitem[\citeproctext]{ref-ghosh_tourism_2021}
Ghosh, S., \& Mitra, S. K. (2021). Tourism and inequality: {A} relook on
the {Kuznets} curve. \emph{Tourism Management}, \emph{83}, 104255.
\url{https://doi.org/10.1016/j.tourman.2020.104255}

\bibitem[\citeproctext]{ref-golden_economic_2014}
Golden, C. D., Bonds, M. H., Brashares, J. S., Rodolph Rasolofoniaina,
B. J., \& Kremen, C. (2014). Economic valuation of subsistence harvest
of wildlife in {Madagascar}. \emph{Conservation Biology}, \emph{28}(1),
234--243. https://doi.org/\url{https://doi.org/10.1111/cobi.12174}

\bibitem[\citeproctext]{ref-golden_rainforest_2012}
Golden, C. D., Rasolofoniaina, B. R., Anjaranirina, E. G., Nicolas, L.,
Ravaoliny, L., \& Kremen, C. (2012). \emph{Rainforest pharmacopeia in
{Madagascar} provides high value for current local and prospective
global uses}.

\bibitem[\citeproctext]{ref-gorgen_efficient_2022}
Görgen, D. A., \& Bhandari, O. P. (2022). \emph{Efficient {Monitoring}
of {Global} {Biodiversity} {Portfolios}}.
\url{https://doi.org/10.32614/CRAN.package.mapme.biodiversity}

\bibitem[\citeproctext]{ref-hansen2013}
Hansen, M. C., Potapov, P. V., Moore, R., Hancher, M., Turubanova, S.
A., Tyukavina, A., Thau, D., Stehman, S. V., Goetz, S. J., Loveland, T.
R., \& others. (2013). High-resolution global maps of 21st-century
forest cover change. \emph{Science}, \emph{342}(6160), 850853.
\url{http://www.sciencemag.org/content/342/6160/850.short}

\bibitem[\citeproctext]{ref-hanson_leviathans_2021}
Hanson, J. K., \& Sigman, R. (2021). Leviathan's {Latent} {Dimensions}:
{Measuring} {State} {Capacity} for {Comparative} {Political} {Research}.
\emph{The Journal of Politics}, \emph{83}(4), 1495--1510.
\url{https://doi.org/10.1086/715066}

\bibitem[\citeproctext]{ref-ho2007}
Ho, D. E., Imai, K., King, G., \& Stuart, E. A. (2007). Matching as
Nonparametric Preprocessing for Reducing Model Dependence in Parametric
Causal Inference. \emph{Political Analysis}, \emph{15}(3), 199236.
\url{https://doi.org/10.1093/pan/mpl013}

\bibitem[\citeproctext]{ref-hunermund_causal_2023}
Hünermund, P., \& Bareinboim, E. (2023). Causal inference and data
fusion in econometrics. \emph{The Econometrics Journal}, utad008.

\bibitem[\citeproctext]{ref-imbens2020}
Imbens, G. W. (2020). Potential outcome and directed acyclic graph
approaches to causality: Relevance for empirical practice in economics.
\emph{Journal of Economic Literature}, \emph{58}(4), 11291179.

\bibitem[\citeproctext]{ref-imbens_causal_2024}
Imbens, G. W. (2024). Causal {Inference} in the {Social} {Sciences}.
\emph{Annual Review of Statistics and Its Application}, \emph{11}(Volume
11, 2024), 123--152.
\url{https://doi.org/10.1146/annurev-statistics-033121-114601}

\bibitem[\citeproctext]{ref-instat_resultats_2020}
INSTAT. (2020a). \emph{Résultats globaux du troisième recensement
général de la population et de l'habitation de 2018 à {Madagascar}
({RGPH}-3)}. Institut national de la statistique.

\bibitem[\citeproctext]{ref-instat2020}
INSTAT. (2020b). \emph{Résultats globaux du troisième recensement
général de la population et de l'habitation de 2018 à madagascar
(RGPH-3)}. Institut national de la statistique.

\bibitem[\citeproctext]{ref-joppa2010}
Joppa, L., \& Pfaff, A. (2010). Reassessing the forest impacts of
protection: The challenge of nonrandom location and a corrective method.
\emph{Annals of the New York Academy of Sciences}, \emph{1185}(1),
135149. \url{https://doi.org/10.1111/j.1749-6632.2009.05162.x}

\bibitem[\citeproctext]{ref-jung_evidence_2019}
Jung, S., Liao, C., Agrawal, A., \& Brown, D. G. (2019). Evidence on
{Wealth}-{Improving} {Effects} of {Forest} {Concessions} in {Liberia}.
\emph{Journal of the Association of Environmental and Resource
Economists}, \emph{6}(5), 961--1000.
\url{https://doi.org/10.1086/704614}

\bibitem[\citeproctext]{ref-kandel_protected_2022}
Kandel, P., Pandit, R., White, B., \& Polyakov, M. (2022). Do protected
areas increase household income? {Evidence} from a {Meta}-{Analysis}.
\emph{World Development}, \emph{159}, 106024.
\url{https://doi.org/10.1016/j.worlddev.2022.106024}

\bibitem[\citeproctext]{ref-keele_overview_2010}
Keele, L. (2010). \emph{An overview of rbounds: {An} {R} package for
{Rosenbaum} bounds sensitivity analysis with matched data.}

\bibitem[\citeproctext]{ref-lo_bue_gender_2022}
Lo Bue, M. C., Le, T. T. N., Santos Silva, M., \& Sen, K. (2022). Gender
and vulnerable employment in the developing world: {Evidence} from
global microdata. \emph{World Development}, \emph{159}, 106010.
\url{https://doi.org/10.1016/j.worlddev.2022.106010}

\bibitem[\citeproctext]{ref-ma_protected_2020}
Ma, B., Zhang, Y., Hou, Y., \& Wen, Y. (2020). Do {Protected} {Areas}
{Matter}? {A} {Systematic} {Review} of the {Social} and {Ecological}
{Impacts} of the {Establishment} of {Protected} {Areas}. \emph{IJERPH},
\emph{17}(19), 7259. \url{https://doi.org/10.3390/ijerph17197259}

\bibitem[\citeproctext]{ref-mammides_evidence_2020}
Mammides, C. (2020). Evidence from eleven countries in four continents
suggests that protected areas are not associated with higher poverty
rates. \emph{Biological Conservation}, \emph{241}, 108353.
\url{https://www.sciencedirect.com/science/article/pii/S0006320719312777?casa_token=1saHx-9SppkAAAAA:sw9KzbZvzqu2WLub5u-K06mA2kgTygSvTi5AEsjBz0rUm8h3h9SKsdId52pG5VEr4SobaFTfguA}

\bibitem[\citeproctext]{ref-maxwell_area-based_2020}
Maxwell, S. L., Cazalis, V., Dudley, N., Hoffmann, M., Rodrigues, A. S.,
Stolton, S., Visconti, P., Woodley, S., Kingston, N., \& Lewis, E.
(2020). Area-based conservation in the twenty-first century.
\emph{Nature}, \emph{586}(7828), 217--227.

\bibitem[\citeproctext]{ref-mckinnon_what_2016}
McKinnon, M. C., Cheng, S. H., Dupre, S., Edmond, J., Garside, R., Glew,
L., Holland, M. B., Levine, E., Masuda, Y. J., Miller, D. C., Oliveira,
I., Revenaz, J., Roe, D., Shamer, S., Wilkie, D., Wongbusarakum, S., \&
Woodhouse, E. (2016). What are the effects of nature conservation on
human well-being? {A} systematic map of empirical evidence from
developing countries. \emph{Environ Evid}, \emph{5}(1), 8.
\url{https://doi.org/10.1186/s13750-016-0058-7}

\bibitem[\citeproctext]{ref-naidoo_evaluating_2019}
Naidoo, R., Gerkey, D., Hole, D., Pfaff, A., Ellis, A. M., Golden, C.
D., Herrera, D., Johnson, K., Mulligan, M., Ricketts, T. H., \& Fisher,
B. (2019). Evaluating the impacts of protected areas on human well-being
across the developing world. \emph{Science Advances}, \emph{5}(4),
eaav3006. \url{https://doi.org/10.1126/sciadv.aav3006}

\bibitem[\citeproctext]{ref-nasa_jpl_nasadem_2020}
NASA JPL. (2020). \emph{{NASADEM} {Merged} {DEM} {Global} 1 arc second
{V001}}. NASA EOSDIS Land Processes Distributed Active Archive Center.
\url{https://doi.org/10.5067/MEASURES/NASADEM/NASADEM_HGT.001}

\bibitem[\citeproctext]{ref-neudert_global_2017}
Neudert, R., Ganzhorn, J. U., \& Waetzold, F. (2017). Global benefits
and local costs--{The} dilemma of tropical forest conservation: {A}
review of the situation in {Madagascar}. \emph{Environmental
Conservation}, \emph{44}(1), 82--96.

\bibitem[\citeproctext]{ref-poudyal_household_2018}
Poudyal, M., Rakotonarivo, O. S., Razafimanahaka, J. H., Hockley, N., \&
Jones, J. P. G. (2018). Household economy, forest dependency \&
opportunity costs of conservation in eastern rainforests of
{Madagascar}. \emph{Sci Data}, \emph{5}(1), 180225.
\url{https://doi.org/10.1038/sdata.2018.225}

\bibitem[\citeproctext]{ref-poudyal_can_2016}
Poudyal, M., Ramamonjisoa, B. S., Hockley, N., Rakotonarivo, O. S.,
Gibbons, J. M., Mandimbiniaina, R., Rasoamanana, A., \& Jones, J. P.
(2016). Can {REDD}+ social safeguards reach the `right'people? {Lessons}
from {Madagascar}. \emph{Global Environmental Change}, \emph{37},
31--42.
\url{https://www.sciencedirect.com/science/article/pii/S095937801630005X}

\bibitem[\citeproctext]{ref-the_dhs_programicf_standard_2018}
Program/ICF, T. D. (2018). \emph{Standard recode manual for {DHS}-7}.
USAID.

\bibitem[\citeproctext]{ref-riviuxe8re2017}
Rivière, M. (2017). \emph{Les (dé)connexions du développement :
ethno-géographie systémique de l'aide au développement et à la
conservation forestière à Amindrabe, Madagascar} {[}PhD thesis{]}.
\url{https://theses.hal.science/tel-01720849}

\bibitem[\citeproctext]{ref-rosenbaum_central_1983}
Rosenbaum, P. (1983). \emph{The central role of the propensity score in
observational studies for causal effects}.

\bibitem[\citeproctext]{ref-rosenbaum2002}
Rosenbaum, P. (2002). \emph{Observational studies}.

\bibitem[\citeproctext]{ref-roth_whats_2023}
Roth, J., Sant'Anna, P. H. C., Bilinski, A., \& Poe, J. (2023). What's
trending in difference-in-differences? {A} synthesis of the recent
econometrics literature. \emph{Journal of Econometrics}, \emph{235}(2),
2218--2244. \url{https://doi.org/10.1016/j.jeconom.2023.03.008}

\bibitem[\citeproctext]{ref-rubin_estimating_1974}
Rubin, D. B. (1974). Estimating causal effects of treatments in
randomized and nonrandomized studies. \emph{Journal of Educational
Psychology}, \emph{66}(5), 688--701.
\url{https://doi.org/10.1037/h0037350}

\bibitem[\citeproctext]{ref-rustein_dhs_2004}
Rustein, S., \& Kiersten, J. (2004). \emph{The {DHS} {Wealth} {Index}}.
ICF International.

\bibitem[\citeproctext]{ref-schleicher_statistical_2020}
Schleicher, J., Eklund, J., D. Barnes, M., Geldmann, J., Oldekop, J. A.,
\& Jones, J. P. (2020b). Statistical matching for conservation science.
\emph{Conservation Biology}, \emph{34}(3), 538--549.

\bibitem[\citeproctext]{ref-schleicher_statistical_2020-1}
Schleicher, J., Eklund, J., D. Barnes, M., Geldmann, J., Oldekop, J. A.,
\& Jones, J. P. (2020a). Statistical matching for conservation science.
\emph{Conservation Biology}, \emph{34}(3), 538--549.

\bibitem[\citeproctext]{ref-skiles_geographically_2013}
Skiles, M. P., Burgert, C. R., Curtis, S. L., \& Spencer, J. (2013).
Geographically linking population and facility surveys: Methodological
considerations. \emph{Population Health Metrics}, \emph{11}(1), 14.
\url{http://www.biomedcentral.com/content/pdf/1478-7954-11-14.pdf}

\bibitem[\citeproctext]{ref-staveteig_intertemporal_2014}
Staveteig, S., \& Mallick, L. (2014). \emph{Intertemporal comparisons of
poverty and wealth with {DHS} data: {A} harmonized asset index
approach}. ICF International.

\bibitem[\citeproctext]{ref-uchida_agglomeration_2011}
Uchida, H., \& Nelson, A. (2011). Agglomeration {Index}: {Towards} a
{New} {Measure} of {Urban} {Concentration}. \emph{Urbanization and
Development: Multidisciplinary Perspectives}.
\url{https://doi.org/10.1093/acprof:oso/9780199590148.003.0003}

\bibitem[\citeproctext]{ref-vicente-serrano_multiscalar_2010}
Vicente-Serrano, S. M., Beguería, S., \& López-Moreno, J. I. (2010). A
{Multiscalar} {Drought} {Index} {Sensitive} to {Global} {Warming}: {The}
{Standardized} {Precipitation} {Evapotranspiration} {Index}.
\emph{Journal of Climate}, \emph{23}(7), 1696--1718.
\url{https://doi.org/10.1175/2009JCLI2909.1}

\bibitem[\citeproctext]{ref-weiss_global_2018}
Weiss, D. J., Nelson, A., Gibson, H. S., Temperley, W., Peedell, S.,
Lieber, A., Hancher, M., Poyart, E., Belchior, S., \& Fullman, N.
(2018). A global map of travel time to cities to assess inequalities in
accessibility in 2015. \emph{Nature}, \emph{553}(7688), 333--336.

\bibitem[\citeproctext]{ref-wilson2006}
Wilson, K. A., McBride, M. F., Bode, M., \& Possingham, H. P. (2006).
Prioritizing global conservation efforts. \emph{Nature},
\emph{440}(7082), 337340.
\url{https://www.nature.com/articles/nature04366}

\bibitem[\citeproctext]{ref-worldpop_global_2018}
WorldPop, \& CIESIN. (2018). \emph{Global 1km {Population}}. Global High
Resolution Population Denominators Project.
\url{https://doi.org/10.5258/SOTON/WP00647}

\bibitem[\citeproctext]{ref-xuanming_impact_2024}
Xuanming, P., Dossou, T. A. M., Dossou, K. P., \& Alinsato, A. S.
(2024). The impact of tourism development on social welfare in {Africa}:
Quantile regression analysis. \emph{Current Issues in Tourism},
\emph{27}(7), 1159--1172.
\url{https://doi.org/10.1080/13683500.2023.2214351}

\end{CSLReferences}




\end{document}
