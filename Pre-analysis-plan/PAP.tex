% Options for packages loaded elsewhere
\PassOptionsToPackage{unicode}{hyperref}
\PassOptionsToPackage{hyphens}{url}
\PassOptionsToPackage{dvipsnames,svgnames,x11names}{xcolor}
%
\documentclass[
]{article}

\usepackage{amsmath,amssymb}
\usepackage{iftex}
\ifPDFTeX
  \usepackage[T1]{fontenc}
  \usepackage[utf8]{inputenc}
  \usepackage{textcomp} % provide euro and other symbols
\else % if luatex or xetex
  \usepackage{unicode-math}
  \defaultfontfeatures{Scale=MatchLowercase}
  \defaultfontfeatures[\rmfamily]{Ligatures=TeX,Scale=1}
\fi
\usepackage{lmodern}
\ifPDFTeX\else  
    % xetex/luatex font selection
\fi
% Use upquote if available, for straight quotes in verbatim environments
\IfFileExists{upquote.sty}{\usepackage{upquote}}{}
\IfFileExists{microtype.sty}{% use microtype if available
  \usepackage[]{microtype}
  \UseMicrotypeSet[protrusion]{basicmath} % disable protrusion for tt fonts
}{}
\makeatletter
\@ifundefined{KOMAClassName}{% if non-KOMA class
  \IfFileExists{parskip.sty}{%
    \usepackage{parskip}
  }{% else
    \setlength{\parindent}{0pt}
    \setlength{\parskip}{6pt plus 2pt minus 1pt}}
}{% if KOMA class
  \KOMAoptions{parskip=half}}
\makeatother
\usepackage{xcolor}
\setlength{\emergencystretch}{3em} % prevent overfull lines
\setcounter{secnumdepth}{5}
% Make \paragraph and \subparagraph free-standing
\makeatletter
\ifx\paragraph\undefined\else
  \let\oldparagraph\paragraph
  \renewcommand{\paragraph}{
    \@ifstar
      \xxxParagraphStar
      \xxxParagraphNoStar
  }
  \newcommand{\xxxParagraphStar}[1]{\oldparagraph*{#1}\mbox{}}
  \newcommand{\xxxParagraphNoStar}[1]{\oldparagraph{#1}\mbox{}}
\fi
\ifx\subparagraph\undefined\else
  \let\oldsubparagraph\subparagraph
  \renewcommand{\subparagraph}{
    \@ifstar
      \xxxSubParagraphStar
      \xxxSubParagraphNoStar
  }
  \newcommand{\xxxSubParagraphStar}[1]{\oldsubparagraph*{#1}\mbox{}}
  \newcommand{\xxxSubParagraphNoStar}[1]{\oldsubparagraph{#1}\mbox{}}
\fi
\makeatother


\providecommand{\tightlist}{%
  \setlength{\itemsep}{0pt}\setlength{\parskip}{0pt}}\usepackage{longtable,booktabs,array}
\usepackage{calc} % for calculating minipage widths
% Correct order of tables after \paragraph or \subparagraph
\usepackage{etoolbox}
\makeatletter
\patchcmd\longtable{\par}{\if@noskipsec\mbox{}\fi\par}{}{}
\makeatother
% Allow footnotes in longtable head/foot
\IfFileExists{footnotehyper.sty}{\usepackage{footnotehyper}}{\usepackage{footnote}}
\makesavenoteenv{longtable}
\usepackage{graphicx}
\makeatletter
\newsavebox\pandoc@box
\newcommand*\pandocbounded[1]{% scales image to fit in text height/width
  \sbox\pandoc@box{#1}%
  \Gscale@div\@tempa{\textheight}{\dimexpr\ht\pandoc@box+\dp\pandoc@box\relax}%
  \Gscale@div\@tempb{\linewidth}{\wd\pandoc@box}%
  \ifdim\@tempb\p@<\@tempa\p@\let\@tempa\@tempb\fi% select the smaller of both
  \ifdim\@tempa\p@<\p@\scalebox{\@tempa}{\usebox\pandoc@box}%
  \else\usebox{\pandoc@box}%
  \fi%
}
% Set default figure placement to htbp
\def\fps@figure{htbp}
\makeatother
% definitions for citeproc citations
\NewDocumentCommand\citeproctext{}{}
\NewDocumentCommand\citeproc{mm}{%
  \begingroup\def\citeproctext{#2}\cite{#1}\endgroup}
\makeatletter
 % allow citations to break across lines
 \let\@cite@ofmt\@firstofone
 % avoid brackets around text for \cite:
 \def\@biblabel#1{}
 \def\@cite#1#2{{#1\if@tempswa , #2\fi}}
\makeatother
\newlength{\cslhangindent}
\setlength{\cslhangindent}{1.5em}
\newlength{\csllabelwidth}
\setlength{\csllabelwidth}{3em}
\newenvironment{CSLReferences}[2] % #1 hanging-indent, #2 entry-spacing
 {\begin{list}{}{%
  \setlength{\itemindent}{0pt}
  \setlength{\leftmargin}{0pt}
  \setlength{\parsep}{0pt}
  % turn on hanging indent if param 1 is 1
  \ifodd #1
   \setlength{\leftmargin}{\cslhangindent}
   \setlength{\itemindent}{-1\cslhangindent}
  \fi
  % set entry spacing
  \setlength{\itemsep}{#2\baselineskip}}}
 {\end{list}}
\usepackage{calc}
\newcommand{\CSLBlock}[1]{\hfill\break\parbox[t]{\linewidth}{\strut\ignorespaces#1\strut}}
\newcommand{\CSLLeftMargin}[1]{\parbox[t]{\csllabelwidth}{\strut#1\strut}}
\newcommand{\CSLRightInline}[1]{\parbox[t]{\linewidth - \csllabelwidth}{\strut#1\strut}}
\newcommand{\CSLIndent}[1]{\hspace{\cslhangindent}#1}

\usepackage{times} % Police Times New Roman
\usepackage{setspace} % Pour interligne
\onehalfspacing % Interligne de 1.5
\usepackage[colorlinks=true,linkcolor=blue,urlcolor=blue]{hyperref} % Activation des liens cliquable

%Packages pour les figures
\usepackage{graphicx} %Inclusion d'images
\usepackage{float} %Positionnement des figures
\usepackage{caption} %Personnalisation des légendes
\usepackage{subcaption} %Gestion des sous-figures

%Personnalisation des notes de bas de page
\usepackage{footmisc} %Options avancées pour les notes de bas de page
\renewcommand{\footnoterule}{\noindent\rule{0.3\linewidth}{0.4pt}\vspace{0.2cm}} %Ligne de séparation plus courte
\renewcommand{\thefootnote}{\textsuperscript{\arabic{footnote}}} % Exposant pour les numéros de notes
\interfootnotelinepenalty=1000 %Empêcher le découpage des notes sur plusieurs pages


\renewcommand{\maketitle}{
  \begin{center}
    \textbf{Journal of Development Economics} \\
    \textbf{Registered Report Stage 1 : Proposal} \\[1em]
    {\Huge \textbf{Conservation and development: Socioeconomic Impact evaluation of Terrestrial Protected Areas in Madagascar based on large national surveys}} \\[1em]
    \Large Iriana Razafimahenina\footnotemark[1]\footnotemark[2]\footnotemark[3]\footnotemark[4]\footnotemark[5]\footnotemark[6],
    Florent Bédécarrats\footnotemark[5]\footnotemark[6],
    Ingrid Dallmann\footnotemark[7],
    Holimalala Randriamanampisoa\footnotemark[1]\footnotemark[3]\footnotemark[6] \\
  \end{center}
  \footnotetext[1]{\href{https://www.univ-antananarivo.mg}{University of Antananarivo}, Madagascar}
  \footnotetext[2]{\href{https://www.universite-paris-saclay.fr}{University of Paris Saclay}, France}
  \footnotetext[3]{Development Centre for Economic Studies and Research (CERED), Madagascar}
  \footnotetext[4]{\href{https://www.ird.fr}{French National Research Institute for Sustainable Development (IRD)}, Madagascar}
  \footnotetext[5]{\href{https://www.uvsq.fr}{University of Saint Quentin en Yvelines}, France}
  \footnotetext[6]{\href{https://www.umi-source.uvsq.fr/}{UMI - Sustainability and Resilience (SOURCE)}, IRD, France}
  \footnotetext[7]{\href{https://www.afd.fr}{Agence Française de Développement (AFD)}, France}
}
\usepackage{sectsty}
\usepackage{titlesec}
\newcommand{\mycenteredsection}[1]{\section*{\centering #1}}
\usepackage{geometry}
\geometry{margin=1.2in}
\usepackage{ragged2e}
\justifying
\interfootnotelinepenalty=10000
\makeatletter
\@ifpackageloaded{caption}{}{\usepackage{caption}}
\AtBeginDocument{%
\ifdefined\contentsname
  \renewcommand*\contentsname{Table of contents}
\else
  \newcommand\contentsname{Table of contents}
\fi
\ifdefined\listfigurename
  \renewcommand*\listfigurename{List of Figures}
\else
  \newcommand\listfigurename{List of Figures}
\fi
\ifdefined\listtablename
  \renewcommand*\listtablename{List of Tables}
\else
  \newcommand\listtablename{List of Tables}
\fi
\ifdefined\figurename
  \renewcommand*\figurename{Figure}
\else
  \newcommand\figurename{Figure}
\fi
\ifdefined\tablename
  \renewcommand*\tablename{Table}
\else
  \newcommand\tablename{Table}
\fi
}
\@ifpackageloaded{float}{}{\usepackage{float}}
\floatstyle{ruled}
\@ifundefined{c@chapter}{\newfloat{codelisting}{h}{lop}}{\newfloat{codelisting}{h}{lop}[chapter]}
\floatname{codelisting}{Listing}
\newcommand*\listoflistings{\listof{codelisting}{List of Listings}}
\makeatother
\makeatletter
\makeatother
\makeatletter
\@ifpackageloaded{caption}{}{\usepackage{caption}}
\@ifpackageloaded{subcaption}{}{\usepackage{subcaption}}
\makeatother

\usepackage{bookmark}

\IfFileExists{xurl.sty}{\usepackage{xurl}}{} % add URL line breaks if available
\urlstyle{same} % disable monospaced font for URLs
\hypersetup{
  pdftitle={Conservation and development: Socioeconomic Impact evaluation of Terrestrial Protected Areas in Madagascar based on large national surveys},
  pdfauthor={Iriana Razafimahenina; Florent Bédécarrats; Ingrid Dallmann; Holimalala Randriamanampisoa},
  colorlinks=true,
  linkcolor={blue},
  filecolor={Maroon},
  citecolor={Blue},
  urlcolor={Blue},
  pdfcreator={LaTeX via pandoc}}


\title{Conservation and development: Socioeconomic Impact evaluation of
Terrestrial Protected Areas in Madagascar based on large national
surveys}
\author{Iriana
Razafimahenina\footnotemark[1]\footnotemark[2]\footnotemark[3]\footnotemark[4]\footnotemark[5]\footnotemark[6] \and Florent
Bédécarrats\footnotemark[5]\footnotemark[6] \and Ingrid
Dallmann\footnotemark[7] \and Holimalala
Randriamanampisoa\footnotemark[1]\footnotemark[3]\footnotemark[6]}
\date{}

\begin{document}
\maketitle


\mycenteredsection{Date of latest draft: 04/12/2024}

\mycenteredsection{Abstract}

Protected Areas are the most widely used tool for biodiversity
conservation. However, their implementation raises concerns about the
well-being of local populations, especially when they are very poor and
dependent on natural resources, as is the case in Madagascar. This
pre-analysis plan outlines the data, methods, and empirical strategies
used to evaluate the impact of protected areas on local household
well-being and the inequalities among them. Our study focuses on
terrestrial protected areas and relies on Demographic Health Surveys
spanning a 13-years period (2008-2021). We will also use data from the
previous 11 years (1997-2008) to assess whether parallel trends prior to
the study period confirm the validity of the comparisons. The data will
be analyzed using spatio-temporal models, matching, and
difference-in-differences methods.

\textbf{Keywords} : Biodiversity Conservation, Well-Being, Demographic
and Health Surveys, Spatio-Temporal Models, Geospatial impact
evaluation, Madagascar

JEL codes : Q57, I31, C31, Q56, O55

\textbf{Study pre-registration}: On open Science Framework (OSF) with
the title `Conservation and development: socioeconomic impact evaluation
of terrestrial Protected Areas in Madagascar based on large national
surveys' \href{https://osf.io/bgu5n/}{https://osf.io/bgu5n/}

\newpage

\textbf{Proposed timeline}

\begin{longtable}[]{@{}
  >{\raggedright\arraybackslash}p{(\linewidth - 2\tabcolsep) * \real{0.6000}}
  >{\raggedright\arraybackslash}p{(\linewidth - 2\tabcolsep) * \real{0.4000}}@{}}
\toprule\noalign{}
\begin{minipage}[b]{\linewidth}\raggedright
Phases
\end{minipage} & \begin{minipage}[b]{\linewidth}\raggedright
Dates
\end{minipage} \\
\midrule\noalign{}
\endhead
\bottomrule\noalign{}
\endlastfoot
Literature Review, Conceptualization, and writing of the Registered
report & May 2024 - January 2025 \\
Retrieve data from selected sources & February 2025 \\
Data cleaning and analysis & February - March 2025 \\
Writing the scientific article & March 2025 - April 2025 \\
Submission to the journal & April 2025 \\
\end{longtable}

\newpage

\section{Introduction}\label{introduction}

The reconciliation between conservation and development has been a
long-discussed issue within the scientific community (Adams et al.,
2004) , but its importance has grown considerably over the past decade
with the rapid expansion of protected areas (PAs). This issue is
particularly relevant for all 195 COP15 signatory states, which have
committed to increasing protected areas coverage to 30\% of terrestrial
land by 2030.

In theory, protected areas can have significant impacts on local
livelihoods, both positive and negative. They are recognized as an
essential tool for biodiversity conservation (Maxwell et al., 2020), but
their creation can deprive nearby communities of access to natural
resources (gathering, hunting, fishing, and medicinal plants), reduce
the amount of land available and restrict economic activities
(agriculture, livestock, construction) (Kandel et al., 2022).
Conversely, they can be accompanied by compensation measures (local
development projects, cash transfers), generate economic benefits (jobs
in protected areas, tourism), and enhance ecosystem services (increased
water resources, erosion control, fire prevention) (Kandel et al.,
2022).

Despite these ambivalent potential effects, empirical studies that
rigorously assess the impact of protected areas on people's livelihoods
are still rare. Of the 1,043 studies reviewed by McKinnon et al. (2016)
, only 19 used quantitative methods to evaluate impacts on material
living conditions or economic well-being. This meta-analysis shows that
the results of studies vary widely depending on the methods used, the
context studied, and the location. Kandel et al. (2022) have updated and
extended this analysis by focusing on a corpus of 30 quantitative
evaluations that specifically address the impact of protected areas on
household income. They show that protected areas can have a positive
impact on local economies, but that this effect is generally modest and
depends on the local context. This variability in impacts highlights the
importance of conducting context-specific studies using robust
quantitative methods.

Madagascar stands out as a particularly relevant case study for
analyzing the relationship between conservation and socioeconomic
conditions. The country is the poorest in terms of the first target of
the Sustainable Development Goals (SDG 1-1), with the highest proportion
of the population living below the international poverty line in the
world (Conceição (2024), p.~298-99). In 2008, terrestrial protected
areas covered 3.6\% of Madagascar and 9\% of the population lived within
10 km of a protected area. Today, they cover 10.8\% and 28\% of the
population live within 10 km of protected areas\footnotemark[8].
Madagascar is also characterized by a low state capacity (Hanson \&
Sigman, 2021), which makes it difficult to implement conservation and
sustainable development policies and the social measures that should
accompany them. These factors, combined with the high dependence of the
rural population on natural resources, mean that the impacts of
protected areas are potentially different from those observed in less
precarious contexts.

However, empirical studies at the national scale are almost non-existent
for Madagascar. None of the quantitative impact evaluations identified
by McKinnon et al. (2016) covered the country. One of the references
consolidated by Kandel et al. (2022) is a multi-country study that
includes Madagascar, but it is based on an estimate of an aggregate
impact at the commune level and covers only one date. It uses the 1993
census data to match the country's municipalities (Mammides, 2020),
without a before-and-after comparison, and in a context where less than
3~\% of the territory was covered by protected areas, most of which had
been created several decades earlier.

Our contribution to the literature is twofold, both empirical and
methodological. Empirically, this study provides an unprecedented
national analysis, covering 71 protected areas established between 2008
and 2021, to evaluate the socioeconomic impacts of conservation in
contexts of extreme poverty and weak governance. Methodologically, it
articulates the state of the art in econometrics, incorporating recent
developments to adapt these methods to the study of protected areas. The
procedure we propose here could be replicated in other countries,
starting with the 39 countries that have at least three geolocated DHS
surveys. This approach paves the way for a more systematic evaluation of
the impact of protected areas, taking into account the specific context
of each country.y.

\footnotetext[8]{ Calculations by the authors based on the location of the DHS survey clusters. The detailed calculation is provided as supplementary material to the study.}

\section{Research Design}\label{research-design}

\subsection{Hypothesis}\label{hypothesis}

Our first hypothesis concerns the overall impact of protected areas in
the Malagasy context. In their meta-analysis of 30 studies, Kandel et
al. (2022) report a slightly positive average impact, but highlight a
large heterogeneity of results across context. Several parameters are
likely to influence impact, as represented graphically in
\autoref{fig:logic_diagram} in the form of directed acyclic graph
(Hünermund \& Bareinboim, 2023; Imbens, 2024)

\begin{figure}[H]
    \centering %Centre l'image horizontalement sur la page
    \includegraphics[width=0.8\linewidth]{C:/Users/irian/Documents/Statistiques/PA-livelihood-impact-dhs2/Pre-analysis-plan/images/figure 1_WU_21.png} %contrôle de la taille
    \caption{Logic diagram of the theory of change tested in the study }
    \caption*{\textit{Source:Authors}} %légende
    \label{fig:logic_diagram} %référencement dans le texte
\end{figure}

The factors likely to lead to a decline in well-being seem particularly
significant in the Malagasy context, where the population is
predominantly rural and living in extreme poverty (the last assessment
was in 2012, with 80.7\% of the population below the \$2.15 a day
threshold at 2017 PPP). Six studies conducted in Madagascar between 1995
and 2006 estimated the opportunity cost of losing access to protected
areas (slash-and-burn agriculture, hunting, gathering, timber, etc.) at
between USD 39 and 177 per household per year (Neudert et al., 2017).
Golden et al. (2014) estimated that income from hunting accounted for
57~\% of household cash income in areas adjacent to the Makira and
Masoala protected areas. Another survey of people living near Makira
estimated the value of pharmaceutical use at USD 30-44 per year per
household, based on the subsidized price of equivalent treatments in the
malagasy market (Golden et al., 2012).

Several factors that could help improve livelihoods through conservation
appear to be fragile in Madagascar, starting with tourism. Naidoo et
al.~(2019) aggregate data from DHS surveys conducted between 2001 and
2011 in 34 developing countries. Their study is based on matching
households near and far from protected areas, but with no pre-post
conservation comparison. They highlight positive impacts, but only for a
subset of protected areas `with documented tourism'\footnotemark[9].
According to their study, households living near the protected areas
`with tourism'~are 17\% wealthier and 16\% less likely to be poor than
similar households living far from these areas.

\footnotetext[9]{The source used to consider that va PA has ‘documented tourism’ is not reported in Naidoo et al 2019.}

However, tourism in Madagascar's protected areas remains low. According
to data from Madagascar National Parks (MNP), only 7 protected areas
recorded more than 10,000 visitors in 2023 (with a maximum of 30,744 in
Isalo), which is low compared to the average of 356,405 visitors per
year and per PA recorded in 929 PAs worldwide in the global study by
Chung et al. (2018). When new protected areas are created in Madagascar,
compensation mechanisms for local populations remain rare, ineffective
and insufficient (Bertrand et al., 2012; Rivière, 2017). The most
in-depth study on this subject, conducted by Poudyal et al. (2018) with
support from the World Bank, focuses on the Ankeniheny Zahamena Corridor
(CAZ), created in 2015 to connect several existing protected areas. Five
study sites were selected: Two adjacent to the new CAZ protected areas
(one eligible for compensation, the other not), two adjacent to
long-established protected areas, and one far from the forest boundary.
The median cost of the conservation restriction is estimated at USD
2,375 per household per year, representing 27\% to 84\% of the average
annual income. The amounts set aside to compensate beneficiary
households were assessed to be insufficient relative to the losses
incurred, and 50\% of households eligible for compensation received
nothing (Poudyal et al., 2016, 2018).

\begin{itemize}
\tightlist
\item
  \textbf{Hypothesis 1:} Protected areas in Madagascar, by limiting
  access to natural resources, have negative impacts on the well-being
  of nearby households that often exceed the benefits of compensation
  and ecosystem services, with more adverse impacts than in other
  countries.
\end{itemize}

The impact mechanisms represented in \autoref{fig:logic_diagram} are
likely to affect households differently depending on their prior
characteristics. Compensation measures are generally implemented in the
form of projects to promote income-generating activities (agriculture,
livestock, handicrafts) in surrounding communities (Poudyal et al.,
2018). In the context of such development projects, individuals known as
`development brokers' frequently emerge as intermediaries between local
communities and implementing organizations. By mobilizing their social
networks and specific skills, these brokers manage to capture a
disproportionate share of the benefits of interventions, whether in form
of income or access to exclusive opportunities. This dynamic can
reinforce pre-existing inequalities within communities, limiting the
access of the most vulnerable households to the expected benefits of
compensation programs.

Although tourism development is often presented as an opportunity for
economic growth, it also tends to exacerbate socioeconomic inequalities,
particularly in developing countries. Adeniyi et al. (2024) show that in
Southern Africa, tourism can initially exacerbate inequalities by
concentrating benefits in the most attractive regions, while leaving
marginalized communities out of the economic benefits. According to
Ghosh and Mitra (2021) the relationship between tourism and inequality
follows an inverted Kuznets curve dynamic in developing countries, when
tourism remains moderate, its growth reduces binequalities, but when
tourism becomes massive, further expansion worsens inequalities.
Finally, Xuanming et al. (2024) point out that while tourism helps to
improve certain socioeconomic indicators, it can also generate
inflationary pressures and strain local resources, particularly
affecting the most vulnerable households.

\textbf{Hypothesis 2}: Protected areas exacerbate economic inequalities
among neighboring communities by creating opportunities that mainly
benefit individuals with a higher educational level or a dominant
position in the community, allowing them access to rents and jobs
related to tourism and associated activities.

Protected areas IUCN status is frequently used to explain differences in
effectiveness between them. For example, Naidoo et al. (2019) show that
multiple-use protected areas (statuses V and VI) tend to have more
beneficial effects than strict areas (statuses I to IV), partly due to
greater flexibility in integrating local needs. Beyond status alone,
governance plays a central role. Eklund et al. (2017) highlight the
importance of transparent and inclusive structures to maximize the
positive effects of protected areas on conservation and social justice.
Similarly, Eklund et al. (2019) call for management approaches to be
adapted to local contexts, with greater involvement of communities in
decision-making processes, to better reconcile conservation and
development objectives.

This diversity is particularly evident in Madagascar. Although governed
by similar formal statutes, protected areas follow different paths
depending on the local context and the way in which they are
implemented. Froger and Méral (2009) show that the early initiatives of
shared governance, gradually introduced with in-depth mediation efforts,
achieved encouraging results by strengthening local community support.
However, from the 2000s onward, the accelerated deployment of management
transfers, driven by quantitative targets, often led to hasty and less
contextually adapted implementations, undermining the effectiveness of
these mechanisms. These experiences demonstrate that, beyond the
protected area status, their establishment period, management approach,
and level of community participation significantly influence their
socio-economic impacts.

\begin{itemize}
\tightlist
\item
  \textbf{Hypothesis 3}: The impacts of protected areas on well-being
  and inequalities are heterogeneous, and some protected areas with good
  levels of local community participation manage to generate greater
  benefits and distribute them more equitably.
\end{itemize}

\subsection{Basic methodological framework / identification
strategy}\label{basic-methodological-framework-identification-strategy}

Our evaluative approach is based on a counterfactual measure that
estimates the causal effect of the treatment, in this case the presence
of protected areas. The counterfactual measure corresponds to a
hypothetical scenario describing what would have happened if the
intervention under study had not taken place. This approach is based on
a comparison between a treatment group (a protected areas) and a control
group (unprotected areas with characteristics very similar to those of
the protected areas). The study thus fits within the framework of
Rubin's causal model Rubin (1974), according to which there are several
hypothetical outcomes depending on exposure to the treatment. To ensure
comparability between groups, matching techniques are used to assign
each unit in the treatment group to a unit with the same observable
characteristics in the control group. Using matching increases the
credibility of research results and reduces endogeneity problems (Ma et
al., 2020).

We subsequently use the difference-in-difference method to estimate the
causal effect attributable to the creation of protected areas. This
method allows us to compare the observed changes in the treatment group
and control group, while compensating for pre-existing differences
between these two groups. By comparing the differences in local
households livelihood before and after the creation of the protected
areas, we isolate the specific effect of the creation of these protected
areas. Matching and difference-in-difference methods are often used
together to reduce selection bias. Several studies have used this
combination of methods to evaluate the impact of conservation on land
use and livelihoods (Ma et al., 2020; Schleicher et al., 2020).

\subsection{Intervention}\label{intervention}

This study evaluates the impact of terrestrial protected areas creation
on rural household well-being between 2008 and 2021. These time frames
were chosen on the basis of the availability of geolocalised data on
household living conditions and coincide with a period of strong
expansion of protected areas in the country, as shown in

\begin{figure}[H]
    \centering %Centre l'image horizontalement sur la page
    \includegraphics[width=0.8\linewidth]{C:/Users/irian/Documents/Statistiques/PA-livelihood-impact-dhs2/Pre-analysis-plan/images/Figure 2_WU_21.png}  %contrôle de la taille
    \caption{Evolution of protected areas in Madagascar and study period}
    \caption*{\textit{Source:Calculations by authors based on data from the Service de la Gouvernance des Aires Protégées (SGAP) of the Ministère de l’environnement et Développement Durable (MEDD)}} %légende
    \label{fig:evol_diagram} %référencement dans le texte
\end{figure}

Protected areas in Madagascar were first created in 1927 under the
French colonial administration and, until the early 2000s, were mainly
characterized by strict conservation (IUCN statuses I, II, and IV). At
the fifth IUCN Parks Summit in Durban in 2003, the Malagasy government
committed to a tripling of protected areas. The declaration led to a
wave of protected area creation, with 28 provisional creation decrees
published between April 2006 and December 2007, and a global decree
bringing the number of new protected areas to 97 in 2008. These decrees
did not designate a manager, leaving it to the Ministry of the
Environment to appoint one. This seems to have been the general
practice, and managers were in place in the majority of the newly
protected areas in the following years. However, it was not until 2015
that a final decree ratified these creations. Some of these protected
areas were the subject of early management transfer decrees, between
2011 and 2015. Some uncertainty remains about the exact date of these
early transfers, and the fact that our study period covers a wider
interval (2008-2021) compensates for this inaccuracy in the start date
of treatment, i.e actual conservation.

\hyperref[Marque-pageux5cux25203]{Table 1} presents the distribution of
protected areas ( in number and area) according to their period of
designation by decree, taking 2008 as the reference year. In the
treatment period (2008 to 2021), 71 protected areas were created
covering 47,282 km².

\section*{Bibliography}\label{bibliography}
\addcontentsline{toc}{section}{Bibliography}

\phantomsection\label{refs}
\begin{CSLReferences}{1}{0}
\bibitem[\citeproctext]{ref-adams_biodiversity_2004}
Adams, William. M., Aveling, R., Brockington, D., Dickson, B., Elliott,
J., Hutton, J., Roe, D., Vira, B., \& Wolmer, W. (2004). Biodiversity
{Conservation} and the {Eradication} of {Poverty}. \emph{Science},
\emph{306}(5699), 1146--1149.
\url{https://doi.org/10.1126/science.1097920}

\bibitem[\citeproctext]{ref-adeniyi_tourism-income_2024}
Adeniyi, O., Adekunle, W., Afolabi, J., \& Kumeka, T. T. (2024).
Tourism-income inequality {Nexus} in {Africa}: Evidence from {SADC}
countries. \emph{Current Issues in Tourism}, \emph{27}(18), 2899--2917.
\url{https://doi.org/10.1080/13683500.2023.2227377}

\bibitem[\citeproctext]{ref-bertrand_contre_2012}
Bertrand, A., Serpantié, G., Randrianarivelo, G., Montagne, P.,
Toillier, A., Karpe, P., Andriambolanoro, D., \& Derycke, M. (2012).
Contre un retour aux barrières~: Quelle place pour la gestion
communautaire dans les nouvelles aires protégées malgaches~? \emph{Les
Cahiers d'Outre-Mer}, \emph{257}(1), 85--123.
\url{https://doi.org/10.4000/com.6493}

\bibitem[\citeproctext]{ref-chung_global_2018}
Chung, M. G., Dietz, T., \& Liu, J. (2018). Global relationships between
biodiversity and nature-based tourism in protected areas.
\emph{Ecosystem Services}, \emph{34}, 11--23.
\url{https://doi.org/10.1016/j.ecoser.2018.09.004}

\bibitem[\citeproctext]{ref-conceicao_breaking_2024}
Conceição, P. (Ed.). (2024). \emph{Breaking the gridlock: {Reimagining}
cooperation in a {Polarized} world}. UNDP.

\bibitem[\citeproctext]{ref-eklund_quality_2017}
Eklund, J., \& Cabeza, M. (2017). Quality of governance and
effectiveness of protected areas: Crucial concepts for conservation
planning. \emph{Annals of the New York Academy of Sciences},
\emph{1399}(1), 27--41. \url{https://doi.org/10.1111/nyas.13284}

\bibitem[\citeproctext]{ref-eklund_what_2019}
Eklund, J., Coad, L., Geldmann, J., \& Cabeza, M. (2019). What
constitutes a useful measure of protected area effectiveness? {A} case
study of management inputs and protected area impacts in {Madagascar}.
\emph{Conservat Sci and Prac}, \emph{1}(10), e107.
\url{https://doi.org/10.1111/csp2.107}

\bibitem[\citeproctext]{ref-froger_temps_2009}
Froger, G., \& Méral, P. (2009). Les temps de la politique
environnementale à {Madagascar} : Entre continuité et bifurcations. In
\emph{Diversité des politiques de développement durable : Temporalités
et durabilités en conflit à {Madagascar}, au {Mali} et au {Mexique}}
(pp. 45--67). Karthala ; GEMDEV.
\url{https://www.documentation.ird.fr/hor/fdi:010048701}

\bibitem[\citeproctext]{ref-ghosh_tourism_2021}
Ghosh, S., \& Mitra, S. K. (2021). Tourism and inequality: {A} relook on
the {Kuznets} curve. \emph{Tourism Management}, \emph{83}, 104255.
\url{https://doi.org/10.1016/j.tourman.2020.104255}

\bibitem[\citeproctext]{ref-golden_economic_2014}
Golden, C. D., Bonds, M. H., Brashares, J. S., Rodolph Rasolofoniaina,
B. J., \& Kremen, C. (2014). Economic valuation of subsistence harvest
of wildlife in {Madagascar}. \emph{Conservation Biology}, \emph{28}(1),
234--243. https://doi.org/\url{https://doi.org/10.1111/cobi.12174}

\bibitem[\citeproctext]{ref-golden_rainforest_2012}
Golden, C. D., Rasolofoniaina, B. R., Anjaranirina, E. G., Nicolas, L.,
Ravaoliny, L., \& Kremen, C. (2012). \emph{Rainforest pharmacopeia in
{Madagascar} provides high value for current local and prospective
global uses}.

\bibitem[\citeproctext]{ref-hanson_leviathans_2021}
Hanson, J. K., \& Sigman, R. (2021). Leviathan's {Latent} {Dimensions}:
{Measuring} {State} {Capacity} for {Comparative} {Political} {Research}.
\emph{The Journal of Politics}, \emph{83}(4), 1495--1510.
\url{https://doi.org/10.1086/715066}

\bibitem[\citeproctext]{ref-hunermund_causal_2023}
Hünermund, P., \& Bareinboim, E. (2023). Causal inference and data
fusion in econometrics. \emph{The Econometrics Journal}, utad008.

\bibitem[\citeproctext]{ref-imbens_causal_2024}
Imbens, G. W. (2024). Causal {Inference} in the {Social} {Sciences}.
\emph{Annual Review of Statistics and Its Application}, \emph{11}(Volume
11, 2024), 123--152.
\url{https://doi.org/10.1146/annurev-statistics-033121-114601}

\bibitem[\citeproctext]{ref-kandel_protected_2022}
Kandel, P., Pandit, R., White, B., \& Polyakov, M. (2022). Do protected
areas increase household income? {Evidence} from a {Meta}-{Analysis}.
\emph{World Development}, \emph{159}, 106024.
\url{https://doi.org/10.1016/j.worlddev.2022.106024}

\bibitem[\citeproctext]{ref-ma_protected_2020}
Ma, B., Zhang, Y., Hou, Y., \& Wen, Y. (2020). Do {Protected} {Areas}
{Matter}? {A} {Systematic} {Review} of the {Social} and {Ecological}
{Impacts} of the {Establishment} of {Protected} {Areas}. \emph{IJERPH},
\emph{17}(19), 7259. \url{https://doi.org/10.3390/ijerph17197259}

\bibitem[\citeproctext]{ref-mammides_evidence_2020}
Mammides, C. (2020). Evidence from eleven countries in four continents
suggests that protected areas are not associated with higher poverty
rates. \emph{Biological Conservation}, \emph{241}, 108353.
\url{https://www.sciencedirect.com/science/article/pii/S0006320719312777?casa_token=1saHx-9SppkAAAAA:sw9KzbZvzqu2WLub5u-K06mA2kgTygSvTi5AEsjBz0rUm8h3h9SKsdId52pG5VEr4SobaFTfguA}

\bibitem[\citeproctext]{ref-maxwell_area-based_2020}
Maxwell, S. L., Cazalis, V., Dudley, N., Hoffmann, M., Rodrigues, A. S.,
Stolton, S., Visconti, P., Woodley, S., Kingston, N., \& Lewis, E.
(2020). Area-based conservation in the twenty-first century.
\emph{Nature}, \emph{586}(7828), 217--227.

\bibitem[\citeproctext]{ref-mckinnon_what_2016}
McKinnon, M. C., Cheng, S. H., Dupre, S., Edmond, J., Garside, R., Glew,
L., Holland, M. B., Levine, E., Masuda, Y. J., Miller, D. C., Oliveira,
I., Revenaz, J., Roe, D., Shamer, S., Wilkie, D., Wongbusarakum, S., \&
Woodhouse, E. (2016). What are the effects of nature conservation on
human well-being? {A} systematic map of empirical evidence from
developing countries. \emph{Environ Evid}, \emph{5}(1), 8.
\url{https://doi.org/10.1186/s13750-016-0058-7}

\bibitem[\citeproctext]{ref-naidoo_evaluating_2019}
Naidoo, R., Gerkey, D., Hole, D., Pfaff, A., Ellis, A. M., Golden, C.
D., Herrera, D., Johnson, K., Mulligan, M., Ricketts, T. H., \& Fisher,
B. (2019). Evaluating the impacts of protected areas on human well-being
across the developing world. \emph{Science Advances}, \emph{5}(4),
eaav3006. \url{https://doi.org/10.1126/sciadv.aav3006}

\bibitem[\citeproctext]{ref-neudert_global_2017}
Neudert, R., Ganzhorn, J. U., \& Waetzold, F. (2017). Global benefits
and local costs--{The} dilemma of tropical forest conservation: {A}
review of the situation in {Madagascar}. \emph{Environmental
Conservation}, \emph{44}(1), 82--96.

\bibitem[\citeproctext]{ref-poudyal_household_2018}
Poudyal, M., Rakotonarivo, O. S., Razafimanahaka, J. H., Hockley, N., \&
Jones, J. P. G. (2018). Household economy, forest dependency \&
opportunity costs of conservation in eastern rainforests of
{Madagascar}. \emph{Sci Data}, \emph{5}(1), 180225.
\url{https://doi.org/10.1038/sdata.2018.225}

\bibitem[\citeproctext]{ref-poudyal_can_2016}
Poudyal, M., Ramamonjisoa, B. S., Hockley, N., Rakotonarivo, O. S.,
Gibbons, J. M., Mandimbiniaina, R., Rasoamanana, A., \& Jones, J. P.
(2016). Can {REDD}+ social safeguards reach the `right'people? {Lessons}
from {Madagascar}. \emph{Global Environmental Change}, \emph{37},
31--42.
\url{https://www.sciencedirect.com/science/article/pii/S095937801630005X}

\bibitem[\citeproctext]{ref-riviuxe8re2017}
Rivière, M. (2017). \emph{Les (dé)connexions du développement :
ethno-géographie systémique de l'aide au développement et à la
conservation forestière à Amindrabe, Madagascar} {[}PhD thesis{]}.
\url{https://theses.hal.science/tel-01720849}

\bibitem[\citeproctext]{ref-rubin_estimating_1974}
Rubin, D. B. (1974). Estimating causal effects of treatments in
randomized and nonrandomized studies. \emph{Journal of Educational
Psychology}, \emph{66}(5), 688--701.
\url{https://doi.org/10.1037/h0037350}

\bibitem[\citeproctext]{ref-schleicher_statistical_2020-1}
Schleicher, J., Eklund, J., D. Barnes, M., Geldmann, J., Oldekop, J. A.,
\& Jones, J. P. (2020). Statistical matching for conservation science.
\emph{Conservation Biology}, \emph{34}(3), 538--549.

\bibitem[\citeproctext]{ref-xuanming_impact_2024}
Xuanming, P., Dossou, T. A. M., Dossou, K. P., \& Alinsato, A. S.
(2024). The impact of tourism development on social welfare in {Africa}:
Quantile regression analysis. \emph{Current Issues in Tourism},
\emph{27}(7), 1159--1172.
\url{https://doi.org/10.1080/13683500.2023.2214351}

\end{CSLReferences}




\end{document}
